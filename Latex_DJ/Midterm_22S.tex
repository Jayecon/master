%%%%%%%%%%%%%%%%%%%%%%%%%%%%%%%%%%%%%%%%%%%%%%%%%%%%%%%%%%%%%%%
%
% Welcome to Overleaf --- just edit your LaTeX on the left,
% and we'll compile it for you on the right. If you open the
% 'Share' menu, you can invite other users to edit at the same
% time. See www.overleaf.com/learn for more info. Enjoy!
%
%%%%%%%%%%%%%%%%%%%%%%%%%%%%%%%%%%%%%%%%%%%%%%%%%%%%%%%%%%%%%%%

% ===============================================
% MATH 790: Real Analysis           Spring 2022
% hw_template.tex
% ===============================================

% -------------------------------------------------------------------------
% The preamble that follows can be ignored. Go on
% down to the section that says "START HERE" 
% -------------------------------------------------------------------------

\documentclass{article}

\usepackage[margin=1in]{geometry} 
\usepackage{amsmath,amsthm,amssymb,hyperref}
\usepackage[hangul]{kotex}
\usepackage[shortlabels]{enumitem}
\usepackage{booktabs, multicol, multirow} % Allows the use of \toprule, \midrule and \bottomrule in tables

\newcommand{\R}{\mathbf{R}}  
\newcommand{\Z}{\mathbf{Z}}
\newcommand{\N}{\mathbf{N}}
\newcommand{\Q}{\mathbf{Q}}

\renewcommand{\labelenumii}{\arabic{enumi}.\arabic{enumii}}
\renewcommand{\labelenumiii}{\arabic{enumi}.\arabic{enumii}.\arabic{enumiii}}
\renewcommand{\labelenumiv}{\arabic{enumi}.\arabic{enumii}.\arabic{enumiii}.\arabic{enumiv}}

\newenvironment{theorem}[2][Theorem]{\begin{trivlist}
\item[\hskip \labelsep {\bfseries #1}\hskip \labelsep {\bfseries #2.}]}{\end{trivlist}}
\newenvironment{lemma}[2][Lemma]{\begin{trivlist}
\item[\hskip \labelsep {\bfseries #1}\hskip \labelsep {\bfseries #2.}]}{\end{trivlist}}
\newenvironment{exercise}[2][Exercise]{\begin{trivlist}
\item[\hskip \labelsep {\bfseries #1}\hskip \labelsep {\bfseries #2.}]}{\end{trivlist}}
\newenvironment{problem}[2][Problem]{\begin{trivlist}
\item[\hskip \labelsep {\bfseries #1}\hskip \labelsep {\bfseries #2.}]}{\end{trivlist}}
\newenvironment{question}[2][Question]{\begin{trivlist}
\item[\hskip \labelsep {\bfseries #1}\hskip \labelsep {\bfseries #2.}]}{\end{trivlist}}
\newenvironment{corollary}[2][Corollary]{\begin{trivlist}
\item[\hskip \labelsep {\bfseries #1}\hskip \labelsep {\bfseries #2.}]}{\end{trivlist}}

\newenvironment{solution}{\begin{proof}[Solution]}{\end{proof}}

\begin{document}

% ------------------------------------------ %
%                 START HERE                  %
% ------------------------------------------ %

\title{중간고사} % Replace with appropriate title
\author{경제정의와 불평등} % Replace "Author's Name" with your name
\date{\today}

\maketitle

% -----------------------------------------------------
% The following two environments (theorem, proof) are
% where you will enter the statement and proof of your
% first problem for this assignment.
%
% In the theorem environment, you can replace the word
% "theorem" in the \begin and \end commands with
% "exercise", "problem", "lemma", etc., depending on
% what you are submitting. 
% -----------------------------------------------------
\begin{enumerate}[{\bf 문제 \arabic*.}]
    \item 제시된 정리 또는 개념에 대한 정의를 쓰고 그 의미를 간단하게 서술하시오.
    \begin{enumerate}
        \item 파레토 효율성
        \item 후생경제학 제 1 정리
        \item 후생경제학 제 2 정리
        \item 중위투표자 정리
        \item 피구-달튼 이전
        \item 균등분배 대등소득
        \item 앳킨슨 지수
    \end{enumerate}
    
    \item A, B 두 사람으로 구성된 사회에서 달성 가능한 효용의 조합은 다음과 같다:
    \[   
    f(u_1,u_2) = 
         \begin{cases}
           u_1 + 2u_2 = 19 &\quad 0 \leq u_1 \leq 1 , u_2 \geq 0 \\
           2u_1 + u_2 = 11 &\quad 1 \leq u_1 \leq 4 , u_2 \geq 0 \\
           4u_1 + u_2 = 19 &\quad 4 \leq u_1  , u_2 \geq 0. 
         \end{cases}
    \]
    이때, 사회적 의사결정이 공리주의, 평등주의($SW=u_1 \times u_2$), 롤즈주의인 경우 사회가 선택하는 효용수준을 그림을 통해 각각 설명하고 정확한 값을 도출 하시오.
    
    \item 표 \ref{tab:scov}\는 세 가지 대안 \{A,B,C\}에 대하여 6명의 선호를 나타낸 것이다. 
        \begin{table}[htbp]
            \centering
            \begin{tabular}{c|c|c|c|c|c|c}
                \toprule
                              & 투표자1 & 투표자2 & 투표자3 & 투표자4 & 투표자5 & 투표자6 \\
                \hline 후보 A & 6 & 4 & 5 & 7 & 0 & 1 \\
                       후보 B & 3 & 3 & 3 & 2 & 7 & 3 \\
                       후보 C & 1 & 3 & 2 & 1 & 3 & 6 \\
                \bottomrule
            \end{tabular}
            \caption{후보에 대한 투표자의 선호}
            \label{tab:scov}
        \end{table}
        \begin{enumerate}
            \item 1인 1표제 하에서 어떤 후보가 승리할 것인가?
            \item 10점의 점수투표제 하에서 전략적 투표가 일어나는 경우와 그 결과를 보이시오.
        \end{enumerate}
        
        \pagebreak
        
    \item 21명으로 구성된 사회에서 A,B,C,D 네 가지 대안에 대하여 \ref{tab:votep} \와 같은 선호를 가지고 있다. 
        \begin{table}[htbp]
            \centering
            \begin{tabular}{cc}
                 3명 & A $\succ$ B $\succ$ C $\succ$ D \\
                 5명 & A $\succ$ C $\succ$ B $\succ$ D \\
                 7명 & B $\succ$ D $\succ$ C $\succ$ A \\
                 6명 & C $\succ$ B $\succ$ D $\succ$ A
            \end{tabular}
            \caption{개인들의 선호}
            \label{tab:votep}
        \end{table}
        \begin{enumerate}
            \item 1인 1표의 다수결 투표하에 어떤 대안이 선택 되는지 설명하시오.
            \item 꽁도세 승자가 있는가? 있다면 어떤 대안인지 설명하시오.
        \end{enumerate}
        
        
    \item HN국은 표 \ref{tab:eitc}\와 같은 근로장려세제를 실시하고 있다. 이때 개인의 여가시간은 최대 20시간으로 가정한다.
        \begin{table}[htbp]
            \centering
            \begin{tabular}{cc}
                 자기소득 & 근로장려금 \\
                 200 미만 & $\text{소득} \times \frac{3}{8}$ \\ 
                 200 - 450 & $75$ \\ 
                 450 - 1000 & $ (100 - \text{소득}) \times \frac{3}{22}$ \\ 
            \end{tabular}
            \caption{근로장려금 산정방식}
            \label{tab:eitc}
        \end{table}
        \begin{enumerate}
            \item 시간당 70의 임금을 받는 사람이 선택가능한 여가와 소득의 조합을 그림으로 나타내라.
            \item 자신이 그린 그림을 이용하여 근로장려세제의 영향을 받지 않는 사람의 선호와 특성에 대하여 설명하시오.
        \end{enumerate}
    
    \item 표 \ref{tab:incomed}는 세 사회의 구성원의 소득을 조사한 자료이다.
        \begin{table}[htbp]
            \centering
            \begin{tabular}{c|c|c|c|c|c}
                \toprule
                   사회 A & 3 & 12 & 30 & 50 & 148  \\
                   사회 B & 12 & 20 & 28 & 48 & 100 \\
                   사회 C & 10 & 12 & 14 & 40 & 120  \\
                \bottomrule
            \end{tabular}
            \caption{소득조사}
            \label{tab:incomed}
        \end{table}
        \begin{enumerate}
            \item 각각의 사회의 로렌츠 곡선을 그리고, 로렌츠 지배 여부를 판단하라.
            \item 각 사회의 지니계수를 각각 구하라.
            \item 공리주의, 평등주의(각 개인소득의 단순 곱), 롤즈주의 각각에서 어떤 사회가 더 나은가?
            \item 상대적 빈곤율을 정의하고 이 관점에서 각각의 사회를 판단하라.
        \end{enumerate}
\end{enumerate}



% ---------------------------------------------------
% Anything after the \end{document} will be ignored by the typesetting.
% ----------------------------------------------------

\end{document}