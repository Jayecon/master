% Table generated by Excel2LaTeX from sheet 'Sheet1'
\begin{table}[htbp]
\centering
\caption{출신지역으로 구분한 자료요약}
\label{tab:knk_sumstat_region}%
\resizebox{\textwidth}{!}{
    \begin{tabular}{c|c|c|c|c|c|c|c|c}
    \hline
    \multirow{2}{*}{환경 변수} & \multirow{2}{*}{시험 구분} & \multirow{2}{*}{환경수준} & \multicolumn{3}{c|}{언어영역} & \multicolumn{3}{c}{외국어영역} \\
    \cline{4-9}          &       & \multicolumn{1}{c|}{} & \multicolumn{1}{c|}{학생수} & \multicolumn{1}{c|}{환경내 비율} & \multicolumn{1}{c|}{수능 응시율} & \multicolumn{1}{c|}{학생수} & \multicolumn{1}{c|}{환경내 비율} & \multicolumn{1}{c}{수능 응시율} \\
    \hline
    \multirow{6}[4]{*}{출신지역} & \multirow{3}[2]{*}{05학년 수능} & 저(농$\cdot$어촌) & 272   & 15.57\% & 72.15\% & 272   & 15.57\% & 72.15\% \\
    &       & 중(중소도시) & 586   & 33.54\% & 89.19\% & 586   & 33.54\% & 89.19\% \\
    &       & 고(대도시) & 889   & 50.89\% & 92.99\% & 889   & 50.89\% & 92.99\% \\
    \cline{2-9}          & \multirow{3}[2]{*}{11학년 수능} & 저(농$/cdot$어촌) & 525   & 13.85\% & 61.33\% & 516   & 13.86\% & 60.28\% \\
    &       & 중(중소도시) & 1438  & 37.93\% & 72.44\% & 1410  & 37.93\% & 71.03\% \\
    &       & 고(대도시) & 1828  & 48.22\% & 73.65\% & 1797  & 48.22\% & 72.40\% \\
    \hline
    \end{tabular}}
\end{table}%
