    \begin{tabular}{cp{7cm}p{7cm}}
        \toprule
        & \textbf{장점} & \textbf{단점} \\
        \midrule
        \textbf{연공급} 
        & 생활보장으로 귀속의식 확대 \newline 연공질서 확립과 사기 유지 \newline 폐쇄적 노동시장에서 용이 \newline 실사가 용이 \newline 성과평가가 곤란한 직무에 적용 가능 
        & 동일노동에 대한 동일임금 실시간곤란 \newline 전문기술인력의 확보 곤란 \newline 능력있는 젊은 종업원의 사기 저하 \newline 인건비 부담 가중 \newline 소극적 근무태도 야기 \\
        \midrule
        \textbf{직무급} 
        & 능력주의 인사풍토 조성 \newline 인건비의 효율성 증대 \newline 개인별 임금차 불만 해소 \newline 동일노동에 대한 동일임금 실현 
        & 절차가 복잡 \newline 학력, 연공주의 풍토에서의 저항 \\
        \midrule
        \textbf{직능급} 
        & 능력주의 임금관리 실현 \newline 유용한 인재의 지속적 보유 \newline 종업원의 성장육구기회 제공 
        & 초과능력에 적용 곤란 \newline 직능평가가 어려움 \newline 적용 직종의 제한 \newline 직무 표준화가 선행되어야 함 \\
        \midrule
        \textbf{성과급 (개인성과급)} 
        & 생산성 향상, 종업원 소득 증대 \newline 감독의 필요성 감소 \newline 인건비 측정 용이 
        & 품질관련 문제 발생 가능성 \newline 종업원의 신기술 도입 저하 \newline 생산기계의 고장 시 종업원 불만 고조 \newline 작업장 내 인간관계 문제 발생 가능성 \\
        \bottomrule
    \end{tabular}
