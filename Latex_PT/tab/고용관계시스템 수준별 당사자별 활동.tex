\centering
\begin{tabular}{|>{\centering\arraybackslash}m{2cm}|>{\centering\arraybackslash}m{2.2cm}|>{\centering\arraybackslash}m{2.2cm}|>{\centering\arraybackslash}m{2.2cm}|>{\centering\arraybackslash}m{2.2cm}|>{\centering\arraybackslash}m{2.2cm}|}
\hline
\textbf{분석수준} & \textbf{피고용인/노조} & \textbf{사용자} & \textbf{정부} & \textbf{경영의사결정} & \textbf{공동결정} \\
\hline
\textbf{사업장/기업수준} & 작업집단, 노조간부 및 노측 위원 & 노무담당, 경영자 & 노동감독관, 조정/중재위원 & 인적자원관리 전략 및 관행 & 사업장/기업수준 단체교섭 \\
\hline
\textbf{산업/직업/지역 수준} & 산별/직업별노조, 전문가 단체, 지역연대 & 사용자단체, 컨설팅회사, 노동시장 중개인 & 산업/직업 규제단체, 직업면허발급기관 & 사용자단체의 전략, 공공부문 경영 & 산별교섭 및 패턴 교섭 \\
\hline
\textbf{국가수준} & 전국단위 노조, 노동관련 사회단체 & 전국단위 사용자조직 & 정부, 법률기관 & 기업지배구조와 고용관계 영향 & 국가단위(노사정위) 교섭 및 산별교섭 결과 조정 \\
\hline
\textbf{국제수준} & 국제노동조직(연합), 노동관련 국제적 NGO & 국제사용자조직, 다국적 기업 & 범정부적 기구(EU, ILO, WTO) & 다국적기업 내 경영정책 확산 & 다국적기업의 교섭 및 협의 \\
\hline
\end{tabular}