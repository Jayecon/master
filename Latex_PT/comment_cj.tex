%----------------------------------------------------------
% PACKAGES AND THEMES
%----------------------------------------------------------
\documentclass[aspectratio=169,xcolor=dvipsnames,handout]{beamer}

\usetheme{Darmstadt}
\usecolortheme{seahorse}
\setbeamercovered{transparent}

\usepackage[hangul]{kotex}
\usepackage{hyperref}
\usepackage{graphicx, array, adjustbox, makecell}
\usepackage{booktabs, multicol, multirow}

% font조정
%\usepackage{fontspec}
%\setmainfont{Times New Roman}
%\setmainhangulfont{NanumGothic}

% 문자열 대체{노사관계론 전용}
\usepackage{newunicodechar}
\newunicodechar{•}{$\cdot$}
\newunicodechar{➔}{$\implies$}
\newunicodechar{∴}{$\therefore$}
\newunicodechar{∵}{$\because$}

%----------------------------------------------------------
% TITLE PAGE
%----------------------------------------------------------
\title{Climate Justice에 대한 논평}
\subtitle{홍현우* 주병기}
\author{오성재}
\institute[CNU]
    {\relax
        한국경제학회 충청지회·한국지역정책학회 추계학술대회
    }
\date{2024년 11월 29일}

%----------------------------------------------------------
\begin{document}
%----------------------------------------------------------

\frame{\titlepage}

\begin{frame}{목차}
    \tableofcontents
\end{frame}

\section{발표문 요약}

\begin{frame}
    \frametitle{문제와 목표}
    \begin{itemize}[<+->]
        \item 문제 정의
        \begin{itemize}
            \item 기후정의의 갈등:
            \begin{itemize}
               \item 개발도상국의 책임을 강조한 역사적 책임성과
               \item 모든 국가의 이익을 보장하는 국제 파레토주의 간의 충돌.
            \end{itemize}
            \item 교토의정서 비판:
            \begin{itemize}
               \item 선진국에 과도한 책임을 부과, 개발도상국과 선진국 간의 불균형 유발.
            \end{itemize}
        \end{itemize}
        \item 연구 목표
        \begin{itemize}
            \item 역사적 책임성과 국제 파레토주의를 경제 모델 내 배분 규칙의 공리로 공식화.
            \item 두 공리의 호환성 검증 및 대체 기후 협약 설계 가능성 탐구.
            \item 기존 교토의정서와 새 배분 규칙 비교.
        \end{itemize}
    \end{itemize}
\end{frame}

\begin{frame}
    \frametitle{결과 및 결론}
    \begin{itemize}[<+->]
        \item 주요 결과
        \begin{itemize}
            \item 국제 파레토주의와 역사적 책임성의 비호환성 확인.
            \item 새로운 배분 규칙 (EPC, HEPC, EPDD 등)은 개발도상국에 유리한 재분배 가능성을 제시.
            \item EPDD 규칙은 국제 파레토주의를 충족하지만 선진국에 불리.
        \end{itemize}
        \item 결론
        \begin{itemize}
            \item 재분배의 필요성:
            \begin{itemize}
                \item 개발도상국의 기후 피해 및 낮은 배출량을 고려한 지원 강조.
            \end{itemize}
            \item 역사적 책임성과 국제 파레토주의를 동시에 충족하는 배분 규칙 설계의 어려움.
            \item 정책적 시사점:
            \begin{itemize}
                \item 기후 피해와 감축 비용을 반영한 공정한 협약 필요.
            \end{itemize}
        \end{itemize}
    \end{itemize}
\end{frame}

\section{제언}%

\begin{frame}
    \frametitle{Development Opprotunities (성장기회)}
    \begin{itemize}[<+->]
        \item 그림에서는 성장기회를 (1인당) GDP와 같은 의미로 사용.
        \begin{itemize}[<+->]
            \item 총생산이 성장의 기회로 재정의되는 과정이 누락됨.
            \item GDP는 성장의 기회가 아닌 결과.
            \item GDP성장률?
        \end{itemize}
        \item $\gamma^A_i$: the values of \textbf{development opportunities} afforded by a unit of emissions allowances with agreement
        \begin{itemize}[<+->]
            \item $d_i$/$\gamma^A_i$: country i’s damage (of the disagreement) measured in equivalent \textbf{units of emissions} under international agreement
            \item $\gamma^A_i = GDP_i/c_i$ and $\gamma^D_i = 0$ 
        \end{itemize}
    \end{itemize}
\end{frame}

\begin{frame}
    \frametitle{내생성}
    \begin{itemize}[<+->]
        \item $d_i$: the damage from climate disasters caused by climate change
        \begin{itemize}
            \item is a function of GDP (e.g. 사대강$\downarrow$, VSL$\uparrow$)
        \end{itemize}
        \item $\gamma^A_i$: the values of development opportunities afforded by a unit of emissions allowances with agreement
        \begin{itemize}
            \item is a function of $a_i$: the cost of abating one unit of pollutant.
        \end{itemize}
        \item $a_i$: the cost of abating one unit of pollutant.
        \begin{itemize}
            \item is a function of GDP
        \end{itemize}
        \item $c_i$: the amount of current and future (BAU) emissions
        \begin{itemize}
            \item is a function of $a_i$
        \end{itemize}
    \end{itemize}
\end{frame}

\begin{frame}
    \frametitle{기타}
    \begin{itemize}[<+->]
        \item 모든 변수들은 시간에 따라 다른 값을 가진다.
            \begin{itemize}[<+->]
                \item 하첨자 t가 생략?
            \end{itemize}
        \item 모든 국가가 아닌 특정 국가집단을 대상으로 했을때는 어떤 룰이 유효한가?
            \begin{itemize}[<+->]
                \item 미중 
                \item G7 or OECD
                \item Emission top 10 등등.
            \end{itemize}

    \end{itemize}
\end{frame}


\section*{}%
\begin{frame}
    \centering
    \huge
    감사합니다.
\end{frame}

%------------------------------------------------
\end{document}
%------------------------------------------------

