%경제적 불평등이 경제성장에 어떤 영향을 주는 가에 대한 궁금증은 지금도 많은 관심을 받고 있다.
%불평등과 경제성장 사이의 관계를 다룬 연구는 1990년대이래 꾸준히 다뤄지고 있지만 아직 이들 두 변수 사이의 단기적 관계에 대해 확정적인 결론은 도출되지 않고 있다.
 %본 연구는 성과의 불평등을 기회의 불평등과 노력의 불평등으로 분해하고, 이들 각각이 경제성장에 미치는 영향을 분석한다.
 %국제 학업성취도 검사 자료를 이용한 분석 결과에 의하면, OECD 가입국 집단에서 기회의 불평등은 경제성장에 부정적인 영향을, 노력의 불평등은 경제성장에 긍정적인 영향을 미친다.
 %반면 OECD 미가입 국가 집단에서는 기회불평등과 노력불평등 모두 경제성장에 유의미한 영향을 미치지 않는다.
 %이는 경제제도가 어느 정도 완비된 국가들에서 기회불평등을 완화하는 사회정책은 경제 내의 공평성뿐만 아니라 효율성도 동시에 향상시킬 수 있음을 시사한다.