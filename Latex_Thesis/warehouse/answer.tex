\begin{itemize}
    \item[\textbf{문1}:] 전체 논문구성은 3개의 논문으로 할 것.
    \item[답1:] 2장은 작년에 논문심사를 받았던 대졸자직업이동경로(GOMS)조사를 이용한 대학입학과 초직임금의 기회불평등으로 하였습니다. 3장은 노동패널자료를 이용한 2017년 주병기 교수님과 저의 논문에 노동패널 최신 자료인 2020년 자료를 반영하여 판올림 하였습니다. 4장은 금번에 심사받은 국제교육불평등과 경제성장간의 관계를 쓴 글로 하였습니다. 1장은 들어가는 글이고, 5장은 맺음말로 하여 총 5개 장으로 구성하였습니다.
    \item[\textbf{문2}:] 회귀분석 결과에서 계수값이 이전 연구에 비해 현저히 크다.
    \item[답2:] 환경의 정의가 달라진 영향으로 생각합니다.강창희 교수님과 진행했던 기존 연구에서는 환경변수로 부친의 학력, 장서수, 성별 셋으로 하여 총 18개 환경 유형을 구성하였습니다. 금번 연구는 부친 및 모친의 학력, 장서수, 가정내 소유물 셋을 이용하여 최대 325개의 환경유형으로 구성되어 이전연구에 비하여 많이 세분화 되었습니다. 환경변수에 대한 설정은 더 고민해보겠습니다.
    \item[\textbf{문3}:] 기호불평등 지수의 의미가 \cite{fng11}에서 기술한 내용에서 벗어나 틀렸다.
    \item[답3:] 해당부분에 대하여 \pageref{eq:theil0-decompose}쪽에서 내용을 고쳤습니다. 타일-0(theil-0)를 이용한 기회의 불평등과 잔여불평등의 분해가 동일한 환경에 속한 개인들이 자신이 속한 환경의 평균성적으로 대체된 성적분포에 기반한 불평등지수와 본인의 성적과 속한 환경의 평균성적의 차이로 만들어진 성적분포에 의한 불평등으로 측정된다고 내용을 바꿨습니다. 
    \item[\textbf{문4}:] 불평등이 경제성장에 주는 영향에 대한 이론적 배경이 추가되어야 한다.
    \item[답4:] \pageref{ans4}쪽에 불완전한 자본시장에서 교육과 경제활동간의 선택에 대한 문제, 중위투표자의 분배정책선택을 주요 이론으로 설명하는 문단을 추가하였습니다.
    \item[\textbf{문5}:] 본 연구에서 사용하는 불평등은 경제성장에 대하여 어떤 이론적 입장인가.
    \item[답5:] \pageref{ans5-1}쪽에 해당내용을 기술하였습니다. ``본 연구는 교육성취에서 기회의 불평등과 노력의 불평등을 해당 국가의 노력에 대한 보상이 얼마나 잘 이루어 지는가에 대한 대리변수로 간주한다.'' 동시에 향후 10-20년 이상의 교육자료가 더 축적된다면 교육의 기회불평등을 인적자본의 비효율적 투자로 고려햐여 분석할 수 있을것이라는 글을 \pageref{ans5-2}쪽의 결론에 추가하였습니다.
    \item[\textbf{문6}:] 회귀분석 모형에서 $OECD \times Ineq + nOECD \times Ineq$ 형태의 OECD 변수 통제를 $Ineq + Ineq \times OECD + OECD$로 바꾸는 것이 자연스럽다.
    \item[답6:] \pageref{ans6}쪽에 해당 내용을도 $OECD$ 가변수(dummy variable)와 $Ineq \times OECD $의 교차항을 넣는 것으로 통제방법을 바꾸는 것으로 기술하였습니다. 회귀분석 결과표 역시 전부 수정하였습니다. 기존의 결과에서는 PISA자료를 사용할 경우 비OECD국가에서 잔여의 불평등이 경제성장에 유의하게 부정적인 영향을 주는 었습니다. 바뀐 결과에서는 잔여불평등이 경제성장에 유의한 영향을 주지 못하는 것으로 바꼈습니다. 나머지 경우는 기존의 결과와 변함이 없습니다. TIMSS를 이용할 경우 기회불평등은 경제성장에 유의하게 부정적인 영향을 주는 반면 OECD국가들에 대하여 잔여불평등은 유의하게 긍정적인 영향을 주었습니다. PISA를 통해 살펴본 결과들은 모두 유의하지 않았습니다. 
\end{itemize}
    