공정한 기회평등은 동일한 능력과 야망을 가진 사람들이 각자 처한 사회경제적 환경과 무관하게 동일한 성취의 전망을 갖게 될 때 성립한다. 반면, 기회불평등은 개인의 의지와 독립적으로 주어진 사회경제적 환경에 따라 성취 전망의 우열이 결정될 때 존재한다고 본다. 본 연구는 기회불평등에 대한 철학적 배경과 실증적 연구방법을 소개하고 국내의 자료를 이용하여 교육 및 경제적 성취에서 기회불평등을 실증적으로 분석한다.
1장에서는 기회불평등에 대한 이론적 배경을 소개한다. 먼저 1970년대부터 이어진 철학적 논의들을 소개하면서 기회의 불평등의 핵심개념은 결과에 대한 개인의 책임 유무를 구분하는 것임을 제시한다. 이를 Lefranc et al.(2008)의 확률지배검증을 통해 기회불평등한 상태를 판별하는 방법을 알아본다. 마지막으로 두 가지 기회불평등지수를 통해 기회불평등의 집단간, 시점간 비교를 가능하게 한다.
2장에서는 1장에서 소개한 실증분석방법을 이용해 우리 사회의 교육성취 및 경제력 획득에서 기회불평등의 양상을 알아본다. 분석결과 대학수학능력점수나 우수한 대학 및 학과를 진학하는 데 있어 부모의 학력이나 소득을 환경으로 할 경우, 심각한 기회불평등이 존재하는 것으로 나타났다. 특히 높은 수능점수나 최상위권 대학 및 학과를 진학하는 성공적인 교육성취에 있어 불리한 환경에 속한 학생들 10명중에 최대 7명이 환경적 요인으로 성공에 실패하는 것을 확인했다. 소득획득에 경우 교육성취보다는 기회불평등도가 현저히 낮지만 해외연구와 비교했을 때, 미국, 이탈리아와 같이 기회불평등이 심각한 국가들 보다는 낮고 기회불평등이 거의 없는 북유럽 국가들에 비해서는 높은 편이었다. 특히 젊은 세대에서 불리한 환경에 놓인 개인이 고소득을 획득하는 것이 더욱 어려워 지고 있음을 확인했다.