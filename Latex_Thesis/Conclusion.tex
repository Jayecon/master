본 연구는 우리 사회의 불평등 연구에서 지배적인 성취의 불평등에서 벗어나 기회에 초점을 맞추는 시도이다.
성취는 개인이 타고난 성별, 인종, 가정배경 등의 환경과 이들과는 무관한 개인의 성취의 의지에 의한 결과물이다.
 이때 개인의 처한 환경은 본인의 선택이나 의지와 무관하므로 이로인한 성취의 차이는 개인에게 책임지울 수 없는 반면 본인의 의지에 의한 성취의 차이는 개인이 책임인 부분이다.
 환경에 의해 야기되는 결과의 불평등이 곧 기회의 불평등이고 이를 받아들일 수 없다는 것이 기회의 불평등이 핵심개념이다.

본연구의 목적은 기회의 불평등 개념을 개인의 삶에 대표적인 두 성취인 교육과 경제에 적용하여 우리 사회가 겪는 기회불평등을 정량화하여 제시하는 것이 이다.

2장에서는 수능점수 대신 출신대학 및 학과를 성취로 하는 2000-2011년간 교육의 기회불평등 추이를 분석하였다.
 우리는 대학입학에서 좋은 환경에 속한 학생들이 평균적으로 좋은 대학에 진학하는 기회불평등이 만연함을 확인하였다.
 특히 대입유형별로 수시전형으로 입학한 학생들 간의 기회불평등 정도가 정시로 입학한 학생들 간의 기회불평등 정도보다 높음을 발견했다.
 초직에서 얻는 소득에 대한 분석을 교육적 성취의 기회불평등에 비해 경제적 성취의 기회불평등이 낮음을 알 수 있었다.
 특히 대입유형별로 상이한 소득획득의 기회불평등의 정도를 확인했다.

3장에서는 1998-2017년간 가구소득의 기회불평등을 알아봤다.
 가구주의 연령을 주요 경제활동연령인 30-50세로 제한할 경우 전반적인 기회불평등도는 하락하는 반면 성공의 기회불평등도는 오히려 증가하는 양상을 확인했다.
 해외 주요국과의 비교에서 한국은 미국, 이탈리아 등에 미치지 못하지만 북유럽 국가들과에 비해서는 확연히 높은 수준의 높은수준의 기회불평등을 겪고 있음을 확인했다.

4장에서는 국제 교육평가자료를 이용하여 1995년부터 2019년까지 각국의 청소년들이 겪은 교육성취의 기회불평등을 측정하였다.
 그리고 측정된 기회불평등 해당국가의 노력에 대한 보상 수준으로 간주하고 기회불평등과 경제성장 간의 관계에 대하여 회귀분석을 시행하였다.
 교육성취의 불평등 측정결과 1인당 국민소득이 높은 국가들의 경우 총불평등이 낮은 수준을 보여 발전된 사회$\cdot$교육제도가 학생들간의 교육성취의 절대적 수준의 격차를 줄이는 것을 확인하였다.
 반면 총불평등에서 기회불평등이 차지하는 비중은 미국, 서유럽들이 높아서 선진국들은 교육성취에서 환경의 중요성의 비중이 매우 높음을 확인했다. 
 한국과 같은 OECD 국가들은 교육성취에 있어 노력에 대한 보상의 지표인 잔여 불평등이 높을 수록 경제성장에 긍정적 영향을 주는 것을 확인하였다.
 
이상의 연구를 통해 연구자가 제시하려는 정책제언은 교육성취에서 기회불평등을 줄이는 적극적인 방법을 실천해야 한다는 것이다.
먼저, 소득획득에서의 기회불평등을 줄이는 정책을 사용하는데는 많은 어려움이 뒤따른다.
기회불평등 측정을 위해 성인들의 과거 환경에 대한 정보를 수집해야 하는 현실적 어려움이 존재한다.
이 경우 기술적으로도 많은 어려움이 뒤따를 뿐만 아니라 정책기구의 과도한 정보수집에 대한 반발과 같은 다양한 갈등이 예상된다.
그리고 기회불평등한 상황에 처한 개인 및 가구가 낮은 소득을 얻고 있다는 점에서 현재 시행되고 있는 성취 결과에 의한 소득재분배 정책이 기회불평등을 일정부분 완화하는 역할을 하고 있다.

반면 청소년의 교육에 대하여 동일한 기회를 주는 실질적 평등을 추구하자는 주장은 사회적 합의에 쉽게 도달할 수 있다.
그리고 이들 미래세대가 최대한 공평한 기회를 보장받고 경제활동에 참여할 수 있게 해줘야 미래의 기회불평등 장기적으로 감소 할 수 있다.
소득불평등과 달리 교육불평등은 성취를 얻고난 이후의 재분배가 불가능하므로 기회불평등한 학생들에 대하여 과감한 해소정책을 시행해야 한다.
구체적으로, 현재 대학입시제도의 경우 현재 시행하는 기회균등전형은 우리의 연구결과에 근거했을때 그 규모가 상당히 미흡하다고 할 수 있다.
상위권 대학을 중심으로 기회균등전형을 대폭 확대해야 실질적인 기회균등 개선에 도움이 된다.
대학입학과 같은 입시 결과의 획득에 앞서 학력 성취를 위한 학습환경에 대한 과감한 지원이 이뤄져야 한다. 
열악한 교육환경에 처한 학생에 대한 학비 감면이나 면제를 넘어서 독서실이나 학원, 온라인 강의와 같은 학교 밖 학습에 대하여 학습바우처의 발행과 같은 실질적인 보충 방안을 적극 고려해야 한다.
학습바우처는 현재 지자체간에 경쟁적으로 발행되고 있는 지역화폐처럼 각 지방교육청의 정책경쟁을 도모할 수 있을 것으로 기대된다.
과학고 및 외고와 같은 특수목적학교들이 실질적으로 우월한 환경의 학생들이 다수 진학한다는 점을 고려하여, 열악한 환경에서 우수한 학생들의 교육성취를 적극 지원하는 특수목적학교도 검토해 볼 수 있다.
과학인재 및 외국어인재 양성이 우리 교육이 추구해야할 특수한 목적이라면 열악한 환경에 처한 학생에 대한 양질의 교육기회를 제공하는 것 역시 교육정책이 추구해야 할 주요 목적이라고 할 수 있다.