\documentclass[handout, 10pt]{beamer}
\usepackage[hangul]{kotex}
\usepackage[T1]{fontenc}

% other packages
\usepackage{natbib}
\usepackage{graphicx,float,pstricks,listings,stackengine,xcolor,calligra}
\usepackage{amsmath,amssymb,latexsym}
\usepackage{booktabs,longtable,multicol,multirow,lscape,rotating}
\usepackage{caption,subcaption}
    \newcommand{\source}[1]{\subcaption*{\raggedright 자료: {#1} } }
\usepackage{threeparttable} % Align column caption, table, and notes
\usepackage{adjustbox} % Shrink stuff
%\usepackage{showframe} % Useful for debugging

\author{오성재}
\title{기회불평등과 경제성장}
%\subtitle{}
%\institute{서울대학교 경제학과 박사과정 \\ 박사학위 논문심사}
\institute{한남대학교 탈메지이 교양학부}
\date{\today}
\usetheme{Darmstadt}
%\usepackage{warehouse/PekingU}

% defs
\def\cmd#1{\texttt{\color{red}\footnotesize $\backslash$#1}}
\def\env#1{\texttt{\color{blue}\footnotesize #1}}
\definecolor{deepblue}{rgb}{0,0,0.5}
\definecolor{deepred}{rgb}{0.6,0,0}
\definecolor{deepgreen}{rgb}{0,0.5,0}
\definecolor{halfgray}{gray}{0.55}

\lstset{
    basicstyle=\ttfamily\small,
    keywordstyle=\bfseries\color{deepblue},
    emphstyle=\ttfamily\color{deepred},    % Custom highlighting style
    stringstyle=\color{deepgreen},
    numbers=left,
    numberstyle=\small\color{halfgray},
    rulesepcolor=\color{red!20!green!20!blue!20},
    frame=shadowbox,
}


\begin{document}

\begin{frame}
    \titlepage
    %\begin{figure}[htpb]
        %\begin{center}
            %\includegraphics[width=0.2\linewidth]{pic/PKU_logo.png}
        %\end{center}
    %\end{figure}
\end{frame}

%\AtBeginSection[]
%{
  %\begin{frame}
    %\frametitle{Table of Contents}
    %\tableofcontents[sectionstyle=show,subsectionstyle=show/shaded/hide,subsubsectionstyle=show/shaded/hide,currentsection]
  %\end{frame}
%}

\begin{frame}
    \frametitle{Table of Contents}
    \tableofcontents[sectionstyle=show,subsectionstyle=show/shaded/hide,subsubsectionstyle=show/shaded/hide]
\end{frame}

\section{문제제기}
\begin{frame}
    \begin{itemize}
        \item 불평등과 경제성장의 단기 관계에 대한 실증연구는 다양한 결과를 제시함.
    \end{itemize}
    \begin{figure}[htpb]
        \begin{center}
            \caption{경제성장, 불평등 그리고 빈곤의 관계}
            \includegraphics[scale=0.35]{fig/triangle_relations.png}
            \source{\cite{cetl21}} 
        \end{center}
    \end{figure}
\end{frame}

\section{선행연구}

\subsection{불평등과 경제성장}
\begin{frame}{한 경제의 불평등이 경제성장에 미치는 효과}
    \begin{itemize}
        \item 이론적 연구
        \begin{itemize}
            \item 물적 자본.
            \item 인적 자본(\cite{gnz93}).
            \item 소득재분배(\cite{anr94}; \cite{pnt94}).
            \item 정치 불안정 등등.
        \end{itemize}
        \item 실증적 연구
        \begin{itemize}
            \item 횡단면 자료.
            \item 국가별 패널자료.
            \item 국가별 패널자료 $+$ 기타 고려사항.
        \end{itemize}
    \end{itemize}
\end{frame}

\begin{frame}{실증연구의 동향}
    \begin{itemize}
        \item 횡단면 자료: \cite{barro91}, \cite{anr94}, \cite{pnt94}
        \begin{itemize}
            \item 불평등은 경제성장에 부의 영향
        \end{itemize}
        \item 국가별 패널자료: \cite{lnz98}, \cite{barro20}, \cite{forbes00}, \cite{bnd03}
        \begin{itemize}
            \item 효과의 방향이 불확실, 부/영/정의 영향
        \end{itemize}
        \item 국가별 패널자료 + 국가별 이질성 및 세부 구성요인:
        \begin{itemize}
            \item \cite{voit05, voit11}: 중상위 불평등, 중하위 불평등
            \item \cite{cc10}: 중/저소득 vs. 고소득 국가 집단
            \item \cite{hetl14}: 단기 vs. 장기
        \end{itemize}
    \end{itemize}
\end{frame}


\subsection{기회불평등}
\begin{frame}{기회불평등과 경제성장}
    \begin{itemize}
        \item \cite{mnr13} : 미국내 주를 대상으로 연구를 진행하여 기회불평등이 경제성장에 부정적인 반면 잔여 불평등은 긍정적임을 보임.
        \begin{itemize}
            \item 미국의 50개 주가 분석대상, 1970-1990년대 말.
            \item 사람들의 자유로운 이동이 가능한 경우 기회불평등의 효과 추정치 편의.
        \end{itemize}
        \item \cite{fetl18} : 기회불평등과 경제성장의 관계에 대하여 연구를 진행. 총불평등과 기회불평등이 경제성장과 부의 관계를 가짐을 확인.
        \begin{itemize}
            \item 전 세계 42개국의 가구조사 및 건강조사 미시자료 이용, 1985-2005년.
            \item 경제적 기회의 핵심변수인 부모의 경제력 변수(부모의 학력, 소득, 재산 등)가 부재한 상태 에서 기회불평등도 측정.
        \end{itemize}
    \end{itemize}
\end{frame}

\begin{frame}{기회불평등과 경제성장}
    \begin{itemize}
        \item \cite{ane20} : 기회불평등과 소득불평등의 교차항을 통해 기회불평등한 국가에서 소득의 불평등이 경제성장과 부의 관계를 가짐을 보임.
        \begin{itemize}
            \item 기회불평등을 세대간 소득$\cdot$교육탄력성으로 측정.
            \item 성취에 대한 기회와 노력의 기여에 대한 구분이 없음.
        \end{itemize}
        \item \cite{kno17} : 교육의 불평등을 기회의 불평등과 잔여 불평등으로 분해. OECD 국가들의 경우 기회의 불평등은 경제성장에 부정적 영향을 주고 잔여 불평등은 경제성장에 긍정적 영향을 미침.
        \begin{itemize}
            \item 국제교육평가자료인 TIMSS를 사용.
            \item 기회불평등의 최소한으로 측정하는 불평등지수 사용.
        \end{itemize}
    \end{itemize}
\end{frame}

\section{연구방법}
\subsection{기회불평등지수}
\begin{frame}{기호}
    \begin{itemize}
        \item  $i$ : 개인, $(i \in \{1,\ldots,N \} )$.
        \item 환경변수 $C_{ic}$ : 개인 $i$의 성취에 영향을 주면서 그 개인이 스스로의 의지로 선택할 수 없는 요인.(부모의 소득, 학력, 인종, 성별 등등.).
        \item 환경 $C_i$ : 개인 $i$가 속한 환경변수$C_{ci}$ 들의 c-터플(c-turple) $C_i = (C_{1i}, \ldots , C_{ci})$. 
        \item 환경유형 $T= \{1, \ldots , t \}$, $t= ||C_1|| \times \cdots \times ||C_c||$ : 가능한 모든 환경의 지표(index).
    \end{itemize}
\end{frame}

\begin{frame}{성취, 환경 그리고 노력}
    \begin{itemize}
        \item 개인 $i$의 성취인 성적점수를 $y_i$, 그가 속한 환경을 $C_i$, 노력을 $e_i$라고 표시하자.
        \item 기회불평등의 관점에서 성취는 개인이 속한 환경 및 환경과 무관한 개인의 노력의 결과라고 정의한다.
        \item 개인 $i$ 성취 $y_i$는 환경 $C_i$ 및 노력 $e_i$에 대하여 다음의 선형함수관계를 가진다고 가정한다.
        \begin{equation}
            \label{eq:ols}
             y_{i} =\beta _0 +  \beta _1 C_{1i} + \cdots + \beta _c C_{ci} + \epsilon _i
        \end{equation}
    \end{itemize}
\end{frame}

\begin{frame}{\cite{fng11}}
    \begin{itemize}
        \item 성취의 불평등을 기회의 불평등과 노력의 불평등으로 분해하기 위해 가산적 분해가능성(additively decomposibility)이 있는 타일-0(Theil-0) 지수를 이용한다.
        \begin{equation}
            \label{eq:theil0}
            T(Y)=\frac{1}{N} \sum_{i=1}^{N} \ln \left(\frac{\bar{y}}{y_{i}}\right)
        \end{equation} 
        \item \cite{fng11}는 식 (\ref{eq:ols})을 식 (\ref{eq:theil0})에 대입하는 방법으로 기회불평등과 잔여불평등을 구분한다.
    \end{itemize}
    \begin{equation}
        \label{eq:theil0-decompose}
        T(Y)=T(C ' \cdot \beta) + T(\epsilon)
    \end{equation}
\end{frame}

\begin{frame}{순수한 노력}
    \begin{itemize}
        \item \cite{betl12}는 \cite{Roemer98}의 순수한 노력(pure effort) 개념을 도입하여 식 (\ref{eq:ols})에서 오차항에 존재하는 환경의 영향을 받는 성취를 찾는다.
        \item 오차항이 환경의 영향을 받는다면 조건부 분산 $\sigma ^2 _c = Var[\epsilon |C_c]$이 환경별로 상이할 것이다.
        \item 잔차의 총분산이 $\sigma ^2 =  \sum _c f_c \sigma ^2 _c$인 점을 이용해 오차항의 이분산성을 해소할 수 있다.
    \end{itemize}
\end{frame}

\begin{frame}{\cite{betl12}}
    \begin{itemize}
        \item  $k = ( 1 / \sum _c f _c \sigma ^2 _c ) ^{-1/2}$라고 한다면, 식 (\ref{eq:ols})을 아래와 같이 바꿀 수 있다. 
        \begin{equation}
            \label{eq:bjork}
            \begin{aligned} Y_{i} &= \beta _0 +  \beta _1 C_{1i} + \cdots + \beta _c C_{ci} + \epsilon_{i} \\ &= \beta _0 +  \beta _1 C_{1i} + \cdots + \beta _c C_{ci} +\epsilon_{i}-\underbrace{\epsilon_{i} / k \sigma_{c}}_{u_{i}}+\underbrace{\epsilon_{i} / k \sigma_{c}}_{u_{i}} \\ &= \beta _0 +  \beta _1 C_{1i} + \cdots + \beta _c C_{ci} +\widetilde{\epsilon}_{i}+u_{i}, \end{aligned}
        \end{equation}
        \item 식 (\ref{eq:theil0-decompose})은 아래와 같이 바꿀 수 있다.. 
        \begin{equation}
            \label{eq:theil0-bjork}
            T(Y)=T(C' \cdot \beta +\widetilde{\epsilon}) + T ( u )
        \end{equation}
        \item 위 식의 우변 첫번째 항은 기회의 불평등을 나타내고 두번째 항은 노력의 불평등이다.
    \end{itemize}
\end{frame}

\subsection{회귀분석모형}
\begin{frame}{회귀분석모형}
   \begin{itemize}
       \item  본 연구에서 우리는 성장 회귀분석(growth regression)에서 일반적으로 사용하는 다음의 실증모형을 추정한다.
       \item 회귀분석은 선형회귀분석(OLS), 패널 고정효과(fixed effect) 모형, 시스템 GMM(\cite{bnb98}) 방법을 적용함으로써 주요 계수들을 추정한다.
       \begin{equation}
            \begin{aligned}
            \ln \left(Y_{i, t^{\prime}}\right)-\ln \left(Y_{i, t}\right)=& \delta_{1} I O P_{i, t}+\delta_{2} I O E_{i, t}+\beta_{1} \ln \left(Y_{i, t}\right) \\
            &+X_{i, t} \beta_{2} +\alpha_{i}+\tau_{t^{\prime}}+u_{i, t^{\prime}}
            \end{aligned}
            \label{eq:regbase}
       \end{equation} 
   \end{itemize} 
\end{frame}
 
\begin{frame}{회귀분석모형의 기호}
   \begin{itemize}
        \item $i$ : 국가, $t$ : 교육기회불평등의 측정연도
        \item $t'$ : TIMSS의 경우 $t'=t+4$, PISA의 경우  $t'=t+3$
        \item $Y_{i,t}$ : 국가 $i$의 연도 $t$현재 1인당 GDP
        \item $IOP_{i,t}$ : 기회불평등 지수, $IOE_{i,t}$ : 노력 불평등의 지수
        \item $X_{i,t}$ : 국가 $i$의 $t$연도 현재 특성변수들(인구 수 및 투자재의 가격)의 벡터 
        \item $\alpha _i$ : 국가 고정효과, $\tau _{t'}$ : 연도 고정효과
   \end{itemize} 
\end{frame}

\section{자료소개}
\begin{frame}{TIMSS}
    \begin{itemize}
        \item TIMSS (Trends in International Mathematics and Science Study)
        \begin{itemize}
            \item 1995, 1999, 2003, 2007, 2011, 2015, 2019년도 조사
            \item 수학/과학 학업(curriculum)성취도 검사
            \item 4학년과 8학년 학생들을 대상으로 시행.
        \end{itemize}
    \end{itemize}
\end{frame}

\begin{frame}{PISA}
    \begin{itemize}
         \item PISA (Programme for International Student Assessment)
        \begin{itemize}
            \item 2000, 2003, 2006, 2009 , 2012, 2015, 2018년도 조사
            \item 읽기/수학/과학 문해력(literacy) 검사
            \item 만 15세 대상으로 시행.
        \end{itemize}
    \end{itemize}
\end{frame}

\begin{frame}{추가 자료설명}
    \begin{itemize}
        \item Penn World Table로부터 국가별 1인당 GDP 및 인구, 투자재의 가격 등 거시 통제변수 구축.
        \item 공통사항
        \begin{itemize}
            \item 학생의 개인/가정/교사/학교 정보 조사.
            \item 각 조사마다 34-77개 국가 참가, 총 98개국의 성적 자료.
            \item 전세계 참여학생의 수학성적을 평균 500 분산 100점으로 정규화.
        \end{itemize}
    \end{itemize}
\end{frame}

\begin{frame}{환경변수}
    \begin{itemize}
    \item 부친 및 모친모의 학력 :
    \begin{itemize}
        \item ISCED Lv에 따라 분류(1-7)를 단순 합산하여 2-14의 값을 취함.
    \end{itemize}
    \item 장서수 :
    \begin{itemize}
        \item (1) 10권 미만 - (5) 200권 초과
    \end{itemize}
    \item 가정내 소유물
    \begin{itemize}
        \item 책상, 학생방, 컴퓨터, 인터넷 등 4개 항목을 더미로 조사하여 합산(0-4)
    \end{itemize}
    \item 최대 325개(=13$\times$5$\times$5)의 독립적인 환경 조합을 구성하여 기회불평등의 비중 계산
    \end{itemize}
\end{frame}

\begin{frame}{PISA와 TIMSS의 자료 차이: 응답률}
    \begin{figure}[htpb]
        \begin{center}
            \includegraphics[scale=0.08]{fig/pnt_response.png}
            \caption{PISA와 TIMSS 환경변수 응답률}
        \end{center}
    \end{figure}
    \begin{itemize}
        \item TIMSS의 경우 학생이 자신의 가정환경을 응답하므로 환경변수를 활용하는 본 연구는 해당자료 사용에서 응답률의 문제를 안고 있음.
    \end{itemize}
\end{frame}

\begin{frame}{PISA와 TIMSS의 자료 차이: 학업성취도 vs. 문해력}
    \begin{figure}[htpb]
        \begin{center}
            \includegraphics[scale=0.1]{fig/bar_pntcompare.png}
            \caption{PISA와 TIMSS 불평등 비교}
        \end{center}
    \end{figure}
    \begin{itemize}
        \item 국가별 성적의 평균을 중심으로 살펴볼때 학업성취도와 문해력은 매우 밀접한 관계를 보임.
        \item 국가별 성적의 분포를 중심으로 살펴보면 교육제도가 잘 갖춰진 국가일수록 두 교육성취는 차이를 보임. 
    \end{itemize}
\end{frame}

\section{주요결과}
\subsection{교육불평등지수}
\begin{frame}
    \begin{figure}[htpb]
        \begin{center}
            \includegraphics<1| handout:1>[scale=0.15]{fig/map_bjtpisa_mean.png}
            \includegraphics<2| handout:0>[scale=0.15]{fig/map_bjtpisa_2000.png}
            \includegraphics<3| handout:0>[scale=0.15]{fig/map_bjtpisa_2003.png}
            \includegraphics<4| handout:0>[scale=0.15]{fig/map_bjtpisa_2006.png}
            \includegraphics<5| handout:0>[scale=0.15]{fig/map_bjtpisa_2009.png}
            \includegraphics<6| handout:0>[scale=0.15]{fig/map_bjtpisa_2012.png}
            \includegraphics<7| handout:0>[scale=0.15]{fig/map_bjtpisa_2015.png}
            \includegraphics<8| handout:0>[scale=0.15]{fig/map_bjtpisa_2018.png}
            \caption{PISA 총불평등}
        \end{center}
    \end{figure}
\end{frame}

\begin{frame}
    \begin{figure}[htpb]
        \begin{center}
            \includegraphics<1| handout:1>[scale=0.15]{fig/map_bjrpisa_mean.png}
            \includegraphics<2| handout:0>[scale=0.15]{fig/map_bjrpisa_2000.png}
            \includegraphics<3| handout:0>[scale=0.15]{fig/map_bjrpisa_2003.png}
            \includegraphics<4| handout:0>[scale=0.15]{fig/map_bjrpisa_2006.png}
            \includegraphics<5| handout:0>[scale=0.15]{fig/map_bjrpisa_2009.png}
            \includegraphics<6| handout:0>[scale=0.15]{fig/map_bjrpisa_2012.png}
            \includegraphics<7| handout:0>[scale=0.15]{fig/map_bjrpisa_2015.png}
            \includegraphics<8| handout:0>[scale=0.15]{fig/map_bjrpisa_2018.png}
            \caption{PISA 기회불평등의 비중}
        \end{center}
    \end{figure}
\end{frame}

\begin{frame}
    \begin{figure}[htpb]
        \begin{center}
            \includegraphics<1| handout:1>[scale=0.15]{fig/map_bjttimss_mean.png}
            \includegraphics<2| handout:0>[scale=0.15]{fig/map_bjttimss_1995.png}
            \includegraphics<3| handout:0>[scale=0.15]{fig/map_bjttimss_1999.png}
            \includegraphics<4| handout:0>[scale=0.15]{fig/map_bjttimss_2003.png}
            \includegraphics<5| handout:0>[scale=0.15]{fig/map_bjttimss_2007.png}
            \includegraphics<6| handout:0>[scale=0.15]{fig/map_bjttimss_2011.png}
            \includegraphics<7| handout:0>[scale=0.15]{fig/map_bjttimss_2015.png}
            \includegraphics<8| handout:0>[scale=0.15]{fig/map_bjttimss_2019.png}
            \caption{TIMSS 총불평등}
        \end{center}
    \end{figure}
\end{frame}

\begin{frame}
    \begin{figure}[htpb]
        \begin{center}
            \includegraphics<1| handout:1>[scale=0.15]{fig/map_bjrtimss_mean.png}
            \includegraphics<2| handout:0>[scale=0.15]{fig/map_bjrtimss_1995.png}
            \includegraphics<3| handout:0>[scale=0.15]{fig/map_bjrtimss_1999.png}
            \includegraphics<4| handout:0>[scale=0.15]{fig/map_bjrtimss_2003.png}
            \includegraphics<5| handout:0>[scale=0.15]{fig/map_bjrtimss_2007.png}
            \includegraphics<6| handout:0>[scale=0.15]{fig/map_bjrtimss_2011.png}
            \includegraphics<7| handout:0>[scale=0.15]{fig/map_bjrtimss_2015.png}
            \includegraphics<8| handout:0>[scale=0.15]{fig/map_bjrtimss_2019.png}
            \caption{TIMSS 기회불평등의 비중}
        \end{center}
    \end{figure}
\end{frame}

\subsection{성장회귀분석}
\begin{frame}
    \begin{table}[htbp]
        \begin{adjustbox}{width=\textwidth, totalheight=\textheight-2\baselineskip,keepaspectratio}
            \begin{threeparttable}
                \begin{table}[htbp]\centering
\def\sym#1{\ifmmode^{#1}\else\(^{#1}\)\fi}
\caption{회귀분석 결과 : PISA, 총불평등 \label{tab:pisa_reg1}}
\resizebox{\textwidth}{!}{
\begin{tabular}{l*{6}{c}}
\hline\hline
                    &\multicolumn{1}{c}{(1)}&\multicolumn{1}{c}{(2)}&\multicolumn{1}{c}{(3)}&\multicolumn{1}{c}{(4)}&\multicolumn{1}{c}{(5)}&\multicolumn{1}{c}{(6)}\\
                    &\multicolumn{1}{c}{OLS}&\multicolumn{1}{c}{FE}&\multicolumn{1}{c}{Sys. GMM}&\multicolumn{1}{c}{OLS}&\multicolumn{1}{c}{FE}&\multicolumn{1}{c}{Sys. GMM}\\
\hline
총불평등          &      -2.729\sym{***}&      -4.562\sym{***}&      -2.892         &      -3.667\sym{***}&      -5.296\sym{***}&      -3.770         \\
                    &     (0.972)         &     (1.301)         &     (2.152)         &     (1.254)         &     (1.501)         &     (4.151)         \\
[1em]
OECD $\times$ 총불평등&                     &                     &                     &       2.135         &       5.359         &       3.211         \\
                    &                     &                     &                     &     (2.207)         &     (3.517)         &     (4.288)         \\
[1em]
OECD              &                     &                     &                     &     -0.0489         &     -0.0667         &      0.0297         \\
                    &                     &                     &                     &    (0.0455)         &    (0.0765)         &     (0.154)         \\
[1em]
ln1인당GDP        &       0.926\sym{***}&       0.589\sym{***}&       0.856\sym{***}&       0.934\sym{***}&       0.602\sym{***}&       0.793\sym{***}\\
                    &    (0.0151)         &    (0.0401)         &    (0.0385)         &    (0.0171)         &    (0.0421)         &    (0.0756)         \\
[1em]
투자재가격        &     -0.0429         &      -0.135\sym{**} &     -0.0638         &     -0.0409         &      -0.155\sym{**} &     -0.0387         \\
                    &    (0.0437)         &    (0.0561)         &     (0.104)         &    (0.0441)         &    (0.0609)         &     (0.110)         \\
[1em]
ln인구            &    -0.00826\sym{**} &      -0.186\sym{**} &     -0.0602         &    -0.00721\sym{**} &      -0.213\sym{**} &      -0.188\sym{**} \\
                    &   (0.00324)         &    (0.0864)         &    (0.0510)         &   (0.00336)         &    (0.0900)         &    (0.0877)         \\
[1em]
Constant            &       0.883\sym{***}&       4.948\sym{***}&       1.851\sym{***}&       0.950\sym{***}&       4.914\sym{***}&       2.767\sym{***}\\
                    &     (0.148)         &     (0.504)         &     (0.412)         &     (0.162)         &     (0.530)         &     (0.765)         \\
\hline
r2                  &       0.981         &       0.805         &                     &       0.980         &       0.805         &                     \\
관측수                   &         358         &         358         &         358         &         334         &         334         &         334         \\
국가수                 &                     &          77         &          77         &                     &          77         &          77         \\
\hline\hline
\multicolumn{7}{l}{\footnotesize  괄호안은 표준오차.}\\
\multicolumn{7}{l}{\footnotesize \sym{*} \(p<0.10\), \sym{**} \(p<0.05\), \sym{***} \(p<0.01\)}\\
\end{tabular}}
\end{table}
            \end{threeparttable}
        \end{adjustbox}
    \end{table}
\end{frame}

\begin{frame}
    \begin{table}[htbp]
        \begin{adjustbox}{width=\textwidth, totalheight=\textheight-2\baselineskip,keepaspectratio}
            \begin{threeparttable}
                \centering
\def\sym#1{\ifmmode^{#1}\else\(^{#1}\)\fi}
\caption{PISA 기회불평등 vs. 노력불평등\label{tab:pisacomp}}
\begin{tabular}{l*{6}{c}}
\toprule
                    &\multicolumn{1}{c}{(1)}&\multicolumn{1}{c}{(2)}&\multicolumn{1}{c}{(3)}&\multicolumn{1}{c}{(4)}&\multicolumn{1}{c}{(5)}&\multicolumn{1}{c}{(6)}\\ &\multicolumn{1}{c}{OLS}&\multicolumn{1}{c}{FE}&\multicolumn{1}{c}{Sys. GMM}&\multicolumn{1}{c}{OLS}&\multicolumn{1}{c}{FE}&\multicolumn{1}{c}{Sys. GMM}\\
\midrule
\addlinespace
기회불평등        &      -2.924         &      -0.375         &       9.860         &                     &                     &                     \\
                    &     [-0.59]         &     [-0.06]         &      [0.87]         &                     &                     &                     \\
\addlinespace
노력불평등        &      -3.371\sym{**} &      -5.716\sym{***}&      -6.400\sym{*}  &                     &                     &                     \\
                    &     [-2.01]         &     [-2.78]         &     [-1.65]         &                     &                     &                     \\
\addlinespace
OECD$\times$기회불평등&                     &                     &                     &       3.270         &      -5.126         &       11.99         \\
                    &                     &                     &                     &      [0.70]         &     [-0.48]         &      [0.79]         \\
\addlinespace
OECD$\times$노력불평등&                     &                     &                     &      -5.862\sym{***}&      -0.599         &      -3.274         \\
                    &                     &                     &                     &     [-2.82]         &     [-0.14]         &     [-0.49]         \\
\addlinespace
비OECD$\times$기회불평등&                     &                     &                     &      -6.396         &       2.069         &       4.608         \\
                    &                     &                     &                     &     [-0.88]         &      [0.29]         &      [0.37]         \\
\addlinespace
비OECD$\times$노력불평등&                     &                     &                     &      -2.254         &      -6.566\sym{***}&      -6.761\sym{*}  \\
                    &                     &                     &                     &     [-1.04]         &     [-2.93]         &     [-1.86]         \\
\midrule
r2                  &       0.980         &       0.803         &                     &       0.980         &       0.805         &                     \\
N                   &         334         &         334         &         334         &         334         &         334         &         334         \\
N\_g                 &                     &          77         &          77         &                     &          77         &          77         \\
\bottomrule
\multicolumn{7}{l}{\footnotesize \textit{t} statistics in brackets}\\
\multicolumn{7}{l}{\footnotesize \sym{*} \(p<0.10\), \sym{**} \(p<0.05\), \sym{***} \(p<0.01\)}\\
\end{tabular}

            \end{threeparttable}
        \end{adjustbox}
    \end{table}
\end{frame}


\begin{frame}
    \begin{table}[htbp]
        \begin{adjustbox}{width=\textwidth, totalheight=\textheight-2\baselineskip,keepaspectratio}
            \begin{threeparttable}
                \begin{table}[htbp]\centering
\def\sym#1{\ifmmode^{#1}\else\(^{#1}\)\fi}
\caption{회귀분석 결과 : TIMSS, 총불평등 \label{tab:timss_reg1}}
\resizebox{\textwidth}{!}{
\begin{tabular}{l*{6}{c}}
\hline\hline
                    &\multicolumn{1}{c}{(1)}&\multicolumn{1}{c}{(2)}&\multicolumn{1}{c}{(3)}&\multicolumn{1}{c}{(4)}&\multicolumn{1}{c}{(5)}&\multicolumn{1}{c}{(6)}\\
                    &\multicolumn{1}{c}{OLS}&\multicolumn{1}{c}{FE}&\multicolumn{1}{c}{Sys. GMM}&\multicolumn{1}{c}{OLS}&\multicolumn{1}{c}{FE}&\multicolumn{1}{c}{Sys. GMM}\\
\hline
총불평등          &      -1.172\sym{*}  &       0.199         &      -4.649\sym{*}  &      -1.220\sym{*}  &      0.0398         &      -3.828\sym{*}  \\
                    &     (0.666)         &     (0.803)         &     (2.427)         &     (0.707)         &     (0.802)         &     (2.256)         \\
[1em]
OECD $\times$ 총불평등&                     &                     &                     &       6.624         &       13.40\sym{**} &       13.31\sym{*}  \\
                    &                     &                     &                     &     (4.072)         &     (6.678)         &     (7.506)         \\
[1em]
OECD              &                     &                     &                     &     -0.0595         &      -0.151         &    -0.00967         \\
                    &                     &                     &                     &    (0.0526)         &    (0.0997)         &     (0.118)         \\
[1em]
ln1인당GDP        &       0.927\sym{***}&       0.570\sym{***}&       0.807\sym{***}&       0.925\sym{***}&       0.570\sym{***}&       0.775\sym{***}\\
                    &    (0.0146)         &    (0.0550)         &    (0.0652)         &    (0.0148)         &    (0.0549)         &    (0.0731)         \\
[1em]
투자재가격        &      0.0640         &      -0.148         &      0.0628         &      0.0599         &      -0.138         &    -0.00241         \\
                    &    (0.0491)         &    (0.0901)         &     (0.105)         &    (0.0551)         &    (0.0898)         &    (0.0921)         \\
[1em]
ln인구            &    -0.00920         &      -0.320\sym{***}&      -0.174\sym{**} &     -0.0112\sym{*}  &      -0.327\sym{***}&      -0.222\sym{***}\\
                    &   (0.00606)         &    (0.0988)         &    (0.0714)         &   (0.00670)         &    (0.0985)         &    (0.0811)         \\
[1em]
Constant            &       0.842\sym{***}&       5.384\sym{***}&       2.518\sym{***}&       0.864\sym{***}&       5.402\sym{***}&       2.918\sym{***}\\
                    &     (0.135)         &     (0.641)         &     (0.670)         &     (0.139)         &     (0.644)         &     (0.726)         \\
\hline
r2                  &       0.974         &       0.820         &                     &       0.975         &       0.824         &                     \\
관측수                   &         238         &         238         &         238         &         238         &         238         &         238         \\
국가수                 &                     &          71         &          71         &                     &          71         &          71         \\
\hline\hline
\multicolumn{7}{l}{\footnotesize 괄호안은 표준오차.}\\
\multicolumn{7}{l}{\footnotesize \sym{*} \(p<0.10\), \sym{**} \(p<0.05\), \sym{***} \(p<0.01\)}\\
\end{tabular}}
\end{table}

            \end{threeparttable}
        \end{adjustbox}
    \end{table}
\end{frame}

\begin{frame}
    \begin{table}[htbp]
        \begin{adjustbox}{width=\textwidth, totalheight=\textheight-2\baselineskip,keepaspectratio}
            \begin{threeparttable}
                \begin{table}[htbp]\centering
\def\sym#1{\ifmmode^{#1}\else\(^{#1}\)\fi}
\caption{회귀분석 결과 : TIMSS, 기회 vs. 잔여불평등 \label{tab:timss_reg2}}
\resizebox{\textwidth}{!}{
\begin{tabular}{l*{6}{c}}
\hline\hline
                    &\multicolumn{1}{c}{(1)}&\multicolumn{1}{c}{(2)}&\multicolumn{1}{c}{(3)}&\multicolumn{1}{c}{(4)}&\multicolumn{1}{c}{(5)}&\multicolumn{1}{c}{(6)}\\
                    &\multicolumn{1}{c}{OLS}&\multicolumn{1}{c}{FE}&\multicolumn{1}{c}{Sys. GMM}&\multicolumn{1}{c}{OLS}&\multicolumn{1}{c}{FE}&\multicolumn{1}{c}{Sys. GMM}\\
\hline
기회불평등        &      -1.750         &      -1.374         &      -16.65\sym{**} &      -1.553         &      -2.401         &      -15.11\sym{*}  \\
                    &     (5.203)         &     (5.169)         &     (7.331)         &     (5.534)         &     (5.291)         &     (7.734)         \\
[1em]
잔여불평등        &      -1.053         &       0.483         &      -2.334         &      -1.184         &       0.475         &      -1.620         \\
                    &     (1.204)         &     (1.222)         &     (2.566)         &     (1.292)         &     (1.231)         &     (2.339)         \\
[1em]
OECD $\times$ 기회불평등&                     &                     &                     &      -6.446         &       13.03         &      -2.922         \\
                    &                     &                     &                     &     (12.79)         &     (20.68)         &     (19.52)         \\
[1em]
OECD $\times$ 잔여불평등&                     &                     &                     &       10.34         &       13.88\sym{*}  &       19.41\sym{**} \\
                    &                     &                     &                     &     (6.320)         &     (8.089)         &     (7.735)         \\
[1em]
OECD              &                     &                     &                     &     -0.0637         &      -0.156         &     -0.0299         \\
                    &                     &                     &                     &    (0.0553)         &     (0.101)         &     (0.114)         \\
[1em]
ln1인당GDP        &       0.928\sym{***}&       0.568\sym{***}&       0.817\sym{***}&       0.925\sym{***}&       0.567\sym{***}&       0.780\sym{***}\\
                    &    (0.0145)         &    (0.0553)         &    (0.0671)         &    (0.0149)         &    (0.0554)         &    (0.0764)         \\
[1em]
투자재가격        &      0.0630         &      -0.149         &      0.0520         &      0.0587         &      -0.138         &     0.00736         \\
                    &    (0.0480)         &    (0.0904)         &    (0.0948)         &    (0.0540)         &    (0.0915)         &    (0.0841)         \\
[1em]
ln인구            &    -0.00923         &      -0.318\sym{***}&      -0.178\sym{**} &     -0.0111\sym{*}  &      -0.325\sym{***}&      -0.224\sym{***}\\
                    &   (0.00608)         &    (0.0993)         &    (0.0731)         &   (0.00668)         &    (0.0993)         &    (0.0824)         \\
[1em]
Constant            &       0.839\sym{***}&       5.393\sym{***}&       2.446\sym{***}&       0.869\sym{***}&       5.419\sym{***}&       2.869\sym{***}\\
                    &     (0.134)         &     (0.644)         &     (0.684)         &     (0.140)         &     (0.648)         &     (0.750)         \\
\hline
r2                  &       0.974         &       0.820         &                     &       0.975         &       0.825         &                     \\
관측수                   &         238         &         238         &         238         &         238         &         238         &         238         \\
국가수                 &                     &          71         &          71         &                     &          71         &          71         \\
\hline\hline
\multicolumn{7}{l}{\footnotesize 괄호안은 표준오차.}\\
\multicolumn{7}{l}{\footnotesize \sym{*} \(p<0.10\), \sym{**} \(p<0.05\), \sym{***} \(p<0.01\)}\\
\end{tabular}}
\end{table}
            \end{threeparttable}
        \end{adjustbox}
    \end{table}
\end{frame}

\subsection{성장회귀분석}
\begin{frame}
    \begin{table}[htbp]
        \begin{adjustbox}{width=\textwidth, totalheight=\textheight-2\baselineskip,keepaspectratio}
            \begin{threeparttable}
                \begin{table}[htbp]\centering
\def\sym#1{\ifmmode^{#1}\else\(^{#1}\)\fi}
\caption{회귀분석 결과 : PISA, 지수비교 \label{tab:pisa_reg_rob1}}
\resizebox{\textwidth}{!}{
\begin{tabular}{l*{8}{c}}
\hline\hline
                    &\multicolumn{1}{c}{(1)}&\multicolumn{1}{c}{(2)}&\multicolumn{1}{c}{(3)}&\multicolumn{1}{c}{(4)}&\multicolumn{1}{c}{(5)}&\multicolumn{1}{c}{(6)}&\multicolumn{1}{c}{(7)}&\multicolumn{1}{c}{(8)}\\
                    &\multicolumn{1}{c}{BJ지수}&\multicolumn{1}{c}{FG지수}&\multicolumn{1}{c}{BJ지수}&\multicolumn{1}{c}{FG지수}&\multicolumn{1}{c}{BJ지수}&\multicolumn{1}{c}{FG지수}&\multicolumn{1}{c}{BJ지수}&\multicolumn{1}{c}{FG지수}\\
\hline
총불평등          &      -2.332         &       -1.913    &      -3.770    &      -3.056   &                     &                     &                     &                     \\
                    &     (4.008)         &      (3.834)    &     (4.151)    &     (3.885)   &                     &                     &                     &                     \\
[1em]
OECD $\times$ 총불평등&                     &                     &       3.211         &        1.839   &                     &                     &                     &                     \\
                    &                     &                     &     (4.288)         &      (4.246)   &                     &                     &                     &                     \\
[1em]
기회불평등        &                     &                     &                     &                     &       9.860         &        20.70         &       4.884         &     13.91                        \\
                    &                     &                     &                     &                     &     (11.29)         &      (14.49)         &     (12.28)         &   (13.06)                        \\
[1em]                                                                                                                                                                                          
잔여불평등        &                     &                     &                     &                     &      -6.400\sym{*}  &       -7.144\sym{*}  &      -6.563         &    -6.774\sym{*}                 \\
                    &                     &                     &                     &                     &     (3.871)         &      (3.673)         &     (4.344)         &   (3.861)                        \\
[1em]
OECD $\times$ 기회불평등&        &        &       &                     &                     &                     &       7.417         &        10.52                     \\
                    &        &        &       &                     &                     &                     &     (15.50)         &      (15.15)                     \\
[1em]                                                                                                                                                     
OECD $\times$ 잔여불평등&        &        &       &                     &                     &                     &       1.302         &       -1.214                     \\
                    &        &        &       &                     &                     &                     &     (7.058)         &      (5.660)                     \\
[1em]
OECD              &                     &                     &      0.0297         &      0.0523         &                     &                     &      0.0326         &      0.0499         \\
                    &                     &                     &     (0.154)         &     (0.149)         &                     &                     &     (0.158)         &     (0.158)         \\
[1em]
ln1인당GDP        &       0.817\sym{***}&       0.819\sym{***}&       0.793\sym{***}&       0.793\sym{***}&       0.812\sym{***}&       0.806\sym{***}&       0.790\sym{***}&       0.789\sym{***}\\
                    &    (0.0633)         &    (0.0629)         &    (0.0756)         &    (0.0746)         &    (0.0655)         &    (0.0715)         &    (0.0754)         &    (0.0813)         \\
[1em]
투자재가격        &      0.0206         &      0.0205         &     -0.0387         &     -0.0359         &      0.0155         &    -0.00123         &     -0.0368         &     -0.0420         \\
                    &     (0.160)         &     (0.159)         &     (0.110)         &     (0.109)         &     (0.162)         &     (0.168)         &     (0.110)         &     (0.111)         \\
[1em]
ln인구            &      -0.184\sym{**} &      -0.183\sym{**} &      -0.188\sym{**} &      -0.187\sym{**} &      -0.192\sym{**} &      -0.214\sym{**} &      -0.195\sym{**} &      -0.209\sym{**} \\
                    &    (0.0907)         &    (0.0909)         &    (0.0877)         &    (0.0881)         &    (0.0904)         &    (0.0917)         &    (0.0881)         &    (0.0892)         \\
[1em]
Constant            &       2.495\sym{***}&       2.475\sym{***}&       2.767\sym{***}&       2.748\sym{***}&       2.584\sym{***}&       2.698\sym{***}&       2.815\sym{***}&       2.858\sym{***}\\
                    &     (0.699)         &     (0.689)         &     (0.765)         &     (0.757)         &     (0.709)         &     (0.756)         &     (0.753)         &     (0.801)         \\
\hline
r2                  &                     &                     &                     &                     &                     &                     &                     &                     \\
관측수                   &         334         &         334         &         334         &         334         &         334         &         334         &         334         &         334         \\
국가수                 &          77         &          77         &          77         &          77         &          77         &          77         &          77         &          77         \\
\hline\hline
\multicolumn{9}{l}{\footnotesize  괄호안은 표준오차.}\\
\multicolumn{9}{l}{\footnotesize \sym{*} \(p<0.10\), \sym{**} \(p<0.05\), \sym{***} \(p<0.01\)}\\
\end{tabular}}
\end{table}

            \end{threeparttable}
        \end{adjustbox}
    \end{table}
\end{frame}

\begin{frame}
    \begin{table}[htbp]
        \begin{adjustbox}{width=\textwidth, totalheight=\textheight-2\baselineskip,keepaspectratio}
            \begin{threeparttable}
                \begin{table}[htbp]\centering
\def\sym#1{\ifmmode^{#1}\else\(^{#1}\)\fi}
\caption{회귀분석 결과 : PISA, 시점비교 \label{tab:pisa_reg_rob2}}
\resizebox{\textwidth}{!}{
\begin{tabular}{l*{12}{c}}
\hline\hline
                    &\multicolumn{1}{c}{(1)}&\multicolumn{1}{c}{(2)}&\multicolumn{1}{c}{(3)}&\multicolumn{1}{c}{(4)}&\multicolumn{1}{c}{(5)}&\multicolumn{1}{c}{(6)}&\multicolumn{1}{c}{(7)}&\multicolumn{1}{c}{(8)}&\multicolumn{1}{c}{(9)}&\multicolumn{1}{c}{(10)}&\multicolumn{1}{c}{(11)}&\multicolumn{1}{c}{(12)}\\
                    &\multicolumn{1}{c}{3년후}&\multicolumn{1}{c}{4년후}&\multicolumn{1}{c}{5년후}&\multicolumn{1}{c}{3년후}&\multicolumn{1}{c}{4년후}&\multicolumn{1}{c}{5년후}&\multicolumn{1}{c}{3년후}&\multicolumn{1}{c}{4년후}&\multicolumn{1}{c}{5년후}&\multicolumn{1}{c}{3년후}&\multicolumn{1}{c}{4년후}&\multicolumn{1}{c}{5년후}\\
\hline
총불평등          &      -2.892         &      -1.984         &      -4.083         &      -4.020\sym{*}  &      -3.470         &      -3.577\sym{*}  &                     &                     &                     &                     &                     &                     \\
                    &     (2.152)         &     (2.502)         &     (2.516)         &     (2.339)         &     (2.248)         &     (2.153)         &                     &                     &                     &                     &                     &                     \\
[1em]
OECD $\times$ 총불평등&                     &                     &                     &       3.964         &       1.484         &      -0.287         &                     &                     &                     &                     &                     &                     \\
                    &                     &                     &                     &     (3.858)         &     (6.428)         &     (7.360)         &                     &                     &                     &                     &                     &                     \\
[1em]
기회불평등        &                     &                     &                     &                     &                     &                     &       9.298         &      -5.667         &       7.392         &       4.866         &      -5.595         &       21.07         \\
                    &                     &                     &                     &                     &                     &                     &     (7.529)         &     (13.02)         &     (16.40)         &     (8.326)         &     (13.84)         &     (14.35)         \\
[1em]
잔여불평등        &                     &                     &                     &                     &                     &                     &      -6.167\sym{**} &      -0.968         &      -6.741\sym{*}  &      -6.258\sym{**} &      -2.934         &      -8.950\sym{**} \\
                    &                     &                     &                     &                     &                     &                     &     (2.527)         &     (4.201)         &     (4.036)         &     (2.561)         &     (3.887)         &     (4.212)         \\
[1em]
OECD $\times$ 기회불평등&                     &                     &                     &                     &                     &                     &                     &                     &                     &       7.251         &      -6.557         &      -41.34\sym{*}  \\
                    &                     &                     &                     &                     &                     &                     &                     &                     &                     &     (12.81)         &     (21.48)         &     (23.90)         \\
[1em]
OECD $\times$ 잔여불평등&                     &                     &                     &                     &                     &                     &                     &                     &                     &       1.725         &       5.055         &       9.925         \\
                    &                     &                     &                     &                     &                     &                     &                     &                     &                     &     (5.663)         &     (11.42)         &     (9.945)         \\
[1em]
OECD              &                     &                     &                     &      0.0414         &       0.107         &       0.163         &                     &                     &                     &      0.0528         &      0.0927         &       0.251         \\
                    &                     &                     &                     &    (0.0958)         &     (0.156)         &     (0.204)         &                     &                     &                     &    (0.0970)         &     (0.159)         &     (0.190)         \\
[1em]
ln1인당GDP        &       0.856\sym{***}&       0.962\sym{***}&       0.754\sym{***}&       0.816\sym{***}&       0.843\sym{***}&       0.721\sym{***}&       0.848\sym{***}&       0.968\sym{***}&       0.741\sym{***}&       0.811\sym{***}&       0.847\sym{***}&       0.681\sym{***}\\
                    &    (0.0385)         &     (0.163)         &     (0.140)         &    (0.0507)         &     (0.105)         &    (0.0992)         &    (0.0395)         &     (0.169)         &     (0.150)         &    (0.0506)         &     (0.108)         &    (0.0934)         \\
[1em]
투자재가격        &     -0.0638         &      -0.177         &      -0.344         &      -0.103         &      -0.207         &      -0.374\sym{*}  &     -0.0587         &      -0.177         &      -0.350         &     -0.0947         &      -0.207         &      -0.390\sym{*}  \\
                    &     (0.104)         &     (0.139)         &     (0.221)         &     (0.101)         &     (0.132)         &     (0.224)         &     (0.105)         &     (0.141)         &     (0.219)         &    (0.1000)         &     (0.134)         &     (0.221)         \\
[1em]
ln인구            &     -0.0602         &      -0.142         &       0.344         &     -0.0975         &      -0.142         &       0.181         &     -0.0699         &      -0.144         &       0.356         &      -0.103         &      -0.144         &       0.135         \\
                    &    (0.0510)         &     (0.102)         &     (0.615)         &    (0.0639)         &    (0.0907)         &     (0.530)         &    (0.0518)         &     (0.105)         &     (0.646)         &    (0.0644)         &    (0.0945)         &     (0.500)         \\
[1em]
Constant            &       1.851\sym{***}&       0.947         &       2.110         &       2.202\sym{***}&       2.106\sym{**} &       2.773\sym{**} &       1.956\sym{***}&       0.899         &       2.209         &       2.257\sym{***}&       2.078\sym{**} &       3.242\sym{***}\\
                    &     (0.412)         &     (1.497)         &     (1.858)         &     (0.544)         &     (0.999)         &     (1.402)         &     (0.422)         &     (1.552)         &     (1.887)         &     (0.540)         &     (1.009)         &     (1.235)         \\
\hline
r2                  &                     &                     &                     &                     &                     &                     &                     &                     &                     &                     &                     &                     \\
관측수                   &         358         &         215         &         155         &         358         &         215         &         155         &         358         &         215         &         155         &         358         &         215         &         155         \\
국가수                 &          77         &          70         &          66         &          77         &          70         &          66         &          77         &          70         &          66         &          77         &          70         &          66         \\
\hline\hline
\multicolumn{13}{l}{\footnotesize 괄호안은 표준오차.}\\
\multicolumn{13}{l}{\footnotesize \sym{*} \(p<0.10\), \sym{**} \(p<0.05\), \sym{***} \(p<0.01\)}\\
\end{tabular}}
\end{table}

            \end{threeparttable}
        \end{adjustbox}
    \end{table}
\end{frame}

\begin{frame}
    \begin{table}[htbp]
        \begin{adjustbox}{width=\textwidth, totalheight=\textheight-2\baselineskip,keepaspectratio}
            \begin{threeparttable}
                \begin{table}[htbp]\centering
\def\sym#1{\ifmmode^{#1}\else\(^{#1}\)\fi}
\caption{회귀분석 결과 : TIMSS, 지수비교 \label{tab:timss_reg_rob1}}
\resizebox{\textwidth}{!}{
\begin{tabular}{l*{8}{c}}
\hline\hline
                    &\multicolumn{1}{c}{(1)}&\multicolumn{1}{c}{(2)}&\multicolumn{1}{c}{(3)}&\multicolumn{1}{c}{(4)}&\multicolumn{1}{c}{(5)}&\multicolumn{1}{c}{(6)}&\multicolumn{1}{c}{(7)}&\multicolumn{1}{c}{(8)}\\
                    &\multicolumn{1}{c}{BJ지수}&\multicolumn{1}{c}{FG지수}&\multicolumn{1}{c}{BJ지수}&\multicolumn{1}{c}{FG지수}&\multicolumn{1}{c}{BJ지수}&\multicolumn{1}{c}{FG지수}&\multicolumn{1}{c}{BJ지수}&\multicolumn{1}{c}{FG지수}\\
\hline
총불평등          &      -4.649\sym{*}  &       -0.792          &      -3.828\sym{*}  &   -0.690           &                     &                     &                     &                     \\
                    &     (2.427)         &      (0.938)         &     (2.256)         &     (0.877)        &                     &                     &                     &                     \\
[1em]
OECD $\times$ 총불평등&                     &                     &       13.31\sym{*}  &        20.44\sym{***}&                  &                     &                     &                     \\
                    &                     &                     &     (7.506)         &      (7.028)          &                     &                     &                     &                     \\
[1em]
기회불평등        &                     &                     &                     &                     &      -16.65\sym{**} &       -14.68        &      -15.11\sym{*}  &    -16.62\sym{*}      \\
                    &                     &                     &                     &                     &     (7.331)         &      (10.62)        &     (7.734)         &   (9.189)            \\
[1em]
잔여불평등        &                     &                     &                     &                     &      -2.334         &      0.316         &      -1.620         &    0.602          \\
                    &                     &                     &                     &                     &     (2.566)         &    (1.295)         &     (2.339)         &  (0.995)        \\
[1em]
OECD $\times$ 기회불평등&                     &                     &                     &                     &                     &                     &      -2.922         & -17.66                            \\
                    &                     &                     &                     &                     &                     &                     &     (19.52)         &     (26.57)                       \\
[1em]                                                                                                                                                                                             
OECD $\times$ 잔여불평등&                     &                     &                     &                     &                     &                     &       19.41\sym{**} & 30.08\sym{***}                    \\
                    &                     &                     &                     &                     &                     &                     &     (7.735)         &     (8.551)                       \\
[1em]
OECD              &                     &                     &    -0.00967         &     -0.0881         &                     &                     &     -0.0299         &     -0.0852         \\
                    &                     &                     &     (0.118)         &     (0.122)         &                     &                     &     (0.114)         &     (0.125)         \\
[1em]
ln1인당GDP        &       0.807\sym{***}&       0.819\sym{***}&       0.775\sym{***}&       0.787\sym{***}&       0.817\sym{***}&       0.826\sym{***}&       0.780\sym{***}&       0.782\sym{***}\\
                    &    (0.0652)         &    (0.0606)         &    (0.0731)         &    (0.0689)         &    (0.0671)         &    (0.0597)         &    (0.0764)         &    (0.0699)         \\
[1em]
투자재가격        &      0.0628         &       0.111         &    -0.00241         &      0.0208         &      0.0520         &       0.114         &     0.00736         &      0.0388         \\
                    &     (0.105)         &     (0.107)         &    (0.0921)         &    (0.0906)         &    (0.0948)         &     (0.101)         &    (0.0841)         &    (0.0856)         \\
[1em]
ln인구            &      -0.174\sym{**} &      -0.160\sym{**} &      -0.222\sym{***}&      -0.214\sym{***}&      -0.178\sym{**} &      -0.157\sym{**} &      -0.224\sym{***}&      -0.214\sym{***}\\
                    &    (0.0714)         &    (0.0687)         &    (0.0811)         &    (0.0807)         &    (0.0731)         &    (0.0688)         &    (0.0824)         &    (0.0820)         \\
[1em]
Constant            &       2.518\sym{***}&       2.281\sym{***}&       2.918\sym{***}&       2.699\sym{***}&       2.446\sym{***}&       2.215\sym{***}&       2.869\sym{***}&       2.745\sym{***}\\
                    &     (0.670)         &     (0.628)         &     (0.726)         &     (0.690)         &     (0.684)         &     (0.622)         &     (0.750)         &     (0.700)         \\
\hline
r2                  &                     &                     &                     &                     &                     &                     &                     &                     \\
관측수                   &         238         &         238         &         238         &         238         &         238         &         238         &         238         &         238         \\
국가수                 &          71         &          71         &          71         &          71         &          71         &          71         &          71         &          71         \\
\hline\hline
\multicolumn{9}{l}{\footnotesize  괄호안은 표준오차.}\\
\multicolumn{9}{l}{\footnotesize \sym{*} \(p<0.10\), \sym{**} \(p<0.05\), \sym{***} \(p<0.01\)}\\
\end{tabular}}
\end{table}

            \end{threeparttable}
        \end{adjustbox}
    \end{table}
\end{frame}

\begin{frame}
    \begin{table}[htbp]
        \begin{adjustbox}{width=\textwidth, totalheight=\textheight-2\baselineskip,keepaspectratio}
            \begin{threeparttable}
                \centering
\def\sym#1{\ifmmode^{#1}\else\(^{#1}\)\fi}
\caption{회귀분석 결과 : TIMSS, 시점비교 }
\label{tab:timss_rob2}
\begin{tabular}{l*{12}{c}}
\toprule
                    &\multicolumn{1}{c}{(1)}&\multicolumn{1}{c}{(2)}&\multicolumn{1}{c}{(3)}&\multicolumn{1}{c}{(4)}&\multicolumn{1}{c}{(5)}&\multicolumn{1}{c}{(6)}&\multicolumn{1}{c}{(7)}&\multicolumn{1}{c}{(8)}&\multicolumn{1}{c}{(9)}&\multicolumn{1}{c}{(10)}&\multicolumn{1}{c}{(11)}&\multicolumn{1}{c}{(12)}\\
                    &\multicolumn{1}{c}{3년후}&\multicolumn{1}{c}{4년후}&\multicolumn{1}{c}{5년후}&\multicolumn{1}{c}{3년후}&\multicolumn{1}{c}{4년후}&\multicolumn{1}{c}{5년후}&\multicolumn{1}{c}{3년후}&\multicolumn{1}{c}{4년후}&\multicolumn{1}{c}{5년후}&\multicolumn{1}{c}{3년후}&\multicolumn{1}{c}{4년후}&\multicolumn{1}{c}{5년후}\\
\midrule
L.ln1인당GDP        &       0.842\sym{***}&       0.813\sym{***}&       0.878\sym{***}&       0.804\sym{***}&       0.757\sym{***}&       0.814\sym{***}&       0.851\sym{***}&       0.818\sym{***}&       0.891\sym{***}&       0.807\sym{***}&       0.759\sym{***}&       0.820\sym{***}\\
                    &     [15.81]         &     [11.78]         &      [9.79]         &     [14.63]         &     [10.22]         &      [8.62]         &     [16.42]         &     [11.88]         &     [10.50]         &     [14.74]         &     [10.14]         &      [9.14]         \\
\addlinespace
L.총불평등          &      -1.737         &      -1.729         &      -3.249         &                     &                     &                     &                     &                     &                     &                     &                     &                     \\
                    &     [-1.24]         &     [-1.22]         &     [-1.43]         &                     &                     &                     &                     &                     &                     &                     &                     &                     \\
\addlinespace
L.OECD $\times$ 총불평등&                     &                     &                     &       12.95\sym{**} &       13.83\sym{***}&       8.768\sym{**} &                     &                     &                     &                     &                     &                     \\
                    &                     &                     &                     &      [2.29]         &      [2.69]         &      [1.98]         &                     &                     &                     &                     &                     &                     \\
\addlinespace
L.비OECD $\times$ 총불평등&                     &                     &                     &      -1.219         &      -1.329         &      -2.979         &                     &                     &                     &                     &                     &                     \\
                    &                     &                     &                     &     [-1.07]         &     [-1.08]         &     [-1.41]         &                     &                     &                     &                     &                     &                     \\
\addlinespace
L.기회불평등        &                     &                     &                     &                     &                     &                     &       4.424         &      -11.42         &       18.32         &                     &                     &                     \\
                    &                     &                     &                     &                     &                     &                     &      [0.52]         &     [-1.50]         &      [1.27]         &                     &                     &                     \\
\addlinespace
L.노력불평등        &                     &                     &                     &                     &                     &                     &      -2.315         &       0.194         &      -7.584\sym{**} &                     &                     &                     \\
                    &                     &                     &                     &                     &                     &                     &     [-1.34]         &      [0.11]         &     [-1.98]         &                     &                     &                     \\
\addlinespace
L.OECD $\times$ 기회불평등&                     &                     &                     &                     &                     &                     &                     &                     &                     &      -3.647         &      -2.318         &       6.455         \\
                    &                     &                     &                     &                     &                     &                     &                     &                     &                     &     [-0.17]         &     [-0.12]         &      [0.23]         \\
\addlinespace
L.OECD $\times$ 노력불평등&                     &                     &                     &                     &                     &                     &                     &                     &                     &       19.52\sym{***}&       17.36\sym{***}&       10.07         \\
                    &                     &                     &                     &                     &                     &                     &                     &                     &                     &      [2.88]         &      [3.46]         &      [1.58]         \\
\addlinespace
L.비OECD $\times$ 기회불평등&                     &                     &                     &                     &                     &                     &                     &                     &                     &       7.574         &      -10.10         &       20.35         \\
                    &                     &                     &                     &                     &                     &                     &                     &                     &                     &      [0.82]         &     [-1.32]         &      [1.31]         \\
\addlinespace
L.비OECD $\times$ 노력불평등&                     &                     &                     &                     &                     &                     &                     &                     &                     &      -2.300         &       0.430         &      -7.702\sym{**} \\
                    &                     &                     &                     &                     &                     &                     &                     &                     &                     &     [-1.46]         &      [0.26]         &     [-2.03]         \\
\addlinespace
L.투자재가격        &      0.0351         &      0.0272         &     -0.0789         &     -0.0558         &     -0.0460         &      -0.122         &      0.0211         &      0.0200         &     -0.0857         &     -0.0687         &     -0.0452         &      -0.108         \\
                    &      [0.32]         &      [0.25]         &     [-0.55]         &     [-0.53]         &     [-0.43]         &     [-0.84]         &      [0.19]         &      [0.19]         &     [-0.59]         &     [-0.65]         &     [-0.42]         &     [-0.71]         \\
\addlinespace
L.ln인구            &      -0.133\sym{***}&      -0.200\sym{**} &      -0.133         &      -0.165\sym{***}&      -0.239\sym{***}&      -0.172         &      -0.129\sym{***}&      -0.202\sym{**} &      -0.120         &      -0.159\sym{***}&      -0.240\sym{***}&      -0.161         \\
                    &     [-3.06]         &     [-2.50]         &     [-1.15]         &     [-3.30]         &     [-2.74]         &     [-1.27]         &     [-3.05]         &     [-2.51]         &     [-1.10]         &     [-3.21]         &     [-2.74]         &     [-1.25]         \\
\addlinespace
Constant            &       1.942\sym{***}&       2.480\sym{***}&       1.842\sym{*}  &       2.345\sym{***}&       3.086\sym{***}&       2.534\sym{**} &       1.844\sym{***}&       2.444\sym{***}&       1.677         &       2.290\sym{***}&       3.068\sym{***}&       2.431\sym{**} \\
                    &      [3.37]         &      [3.24]         &      [1.67]         &      [3.93]         &      [3.78]         &      [2.10]         &      [3.32]         &      [3.19]         &      [1.62]         &      [3.90]         &      [3.73]         &      [2.13]         \\
\midrule
r2                  &                     &                     &                     &                     &                     &                     &                     &                     &                     &                     &                     &                     \\
N                   &         324         &         271         &         166         &         324         &         271         &         166         &         324         &         271         &         166         &         324         &         271         &         166         \\
N\_g                 &          70         &          71         &          65         &          70         &          71         &          65         &          70         &          71         &          65         &          70         &          71         &          65         \\
\bottomrule
\multicolumn{13}{l}{\footnotesize \textit{t} statistics in brackets}\\
\multicolumn{13}{l}{\footnotesize \sym{*} \(p<0.10\), \sym{**} \(p<0.05\), \sym{***} \(p<0.01\)}\\
\end{tabular}
            \end{threeparttable}
        \end{adjustbox}
    \end{table}
\end{frame}

\section{결론}
\begin{frame}
    \begin{itemize}
        \item 본연구는 TIMSS와 PISA 두 가지 국제교육평가자료를 이용해 교육의 총불평등을 계산한 후 이를 기회불평등과 노력불평등으로 분해함.
        \begin{itemize}
            \item 총불평등은 동유럽, 중동 및 남미 등 개도국들이 선진국 국가보다 높은 것으로 나타남.
            \item 총불평등에서 기회불평등이 차지하는 비중은 미국, 서유럽과 같은 선진국가들에서도 심각한 것으로 나타남.
        \end{itemize}
    \end{itemize}
\end{frame}

\begin{frame}
    \begin{itemize}
        \item 상이한 성격의 불평등이 3-4년 뒤 경제성장에 어떤 영향을 주는 가에 대하여 분석한 결과 불평등의 성격과 국가의 발전상태에 따라 상이한 방향으로 작용함을 확인.
        \begin{itemize}
            \item 학업성취도의 노력불평등은 OECD 국가들의 4년뒤 경제성장에 부정적 영향을 줌.
            \item 학업성취도의 기회불평등은 비OECD 국가들의 4년뒤 경제성장에 긍정적 영향을 줌.
            \item 문해도의 기회불평등은 3년뒤 경제성장에 유의하지 않은 긍정적 영향을 줌.
            \item 문해도의 노력불평등은 비OECD 국가들의 경우 3년뒤 경제성장에 부정적 영향을 줌.
        \end{itemize}
    \end{itemize}
\end{frame}

\begin{frame}
    \begin{itemize}
        \item 본 연구의 불평등은 교육의 기회불평등임. 
        \begin{itemize}
            \item 경제성장과 관련하여 소득의 기회불평등이 더 적합하지만 자료의 한계가 존재.
        \end{itemize}
        \item 학업성취도와 문해력의 차이에 대한 추가연구가 필요함.
        \begin{itemize}
            \item 경제성장과 관련하여 두 교육성취가 주는 영향의 방향이 상이함을 확인.
        \end{itemize}
        \item 한국의 경우 학생이 노력을 할 수 있는 최소한의 환경을 갖춰주는 것이 필요함.(예: 무료 자습실, 온라인 강의 바우처 등등.)
    \end{itemize}
\end{frame}

\begin{frame}[allowframebreaks]
    \bibliography{Bibliography.bib}
    \bibliographystyle{apalike}
    % \tiny\bibliographystyle{alpha}
\end{frame}

\begin{frame}
    \begin{center}
        {\Huge 감사합니다.}
    \end{center}
\end{frame}

\end{document}