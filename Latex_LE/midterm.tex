\documentclass[11pt,answers]{exam} %정답지를 만들경우 answers 옵션을 넣을 것.
%----------------------------------------------------------
% PACKAGES AND THEMES
%----------------------------------------------------------
\usepackage{kotex}
\usepackage{lastpage}
\usepackage{setspace}
\usepackage{amsmath}

% "point"와 "points"를 "점"으로 변경
    \renewcommand{\pointname}{점}
    \renewcommand{\points}{\pointname}
% 줄 간격 및 문단 간격 조정
    \setlength{\parskip}{0pt}  % 문단 간 여백 제거
    \setstretch{1.2}  % 줄 간격 조정
% 선택지 번호를 숫자로 변경
    %\renewcommand{\thechoice}{\arabic{choice}}
% 선택지 번호를 한글 자모로 변경하는 매크로 (ㄱ, ㄴ, ㄷ, ㄹ)
    \newcommand{\koreanChoice}{
        \ifcase\numexpr\value{choice}-1\relax ㄱ\or\relax ㄴ\or\relax ㄷ\or\relax
        ㄹ\or\relax ㅁ\or\relax ㅂ\or\relax ㅅ\or\relax ㅇ\or\relax ㅈ\or\relax
        ㅊ\or\relax ㅋ\or\relax ㅌ\or\relax ㅍ\or\relax ㅎ\fi
    }
    \renewcommand{\thechoice}{\koreanChoice}
% part 문제 번호를 한글 자모(ㄱ, ㄴ, ㄷ)로 변경하는 매크로
    \newcommand{\koreanPart}{
        \ifcase\value{part}
            ㄱ\or\relax ㄴ\or\relax ㄷ\or\relax ㄹ\or\relax ㅁ\or\relax
            ㅂ\or\relax ㅅ\or\relax ㅇ\or\relax ㅈ\or\relax
            ㅊ\or\relax ㅋ\or\relax ㅌ\or\relax ㅍ\or\relax ㅎ\fi
    }
    \renewcommand{\thepart}{\koreanPart}
    \renewcommand{\partlabel}{\thepart.}
% 페이지 번호를 하단에 출력
    \firstpagefooter{}{\thepage/\pageref{LastPage} 쪽}{}
    \runningfooter{}{\thepage/\pageref{LastPage} 쪽}{}
%----------------------------------------------------------
\begin{document}
%----------------------------------------------------------

% TITLE
    \title{\relax
        2024년 2학기 중간고사 \\
        \Large
        노사관계의 이론과 실재
    }
    \author{담당교수: 오성재}
    \date{2024년 10월 23일}
    \maketitle

% 학생 이름과 학번 입력란
    \noindent
    성명: \makebox[.3\textwidth]{\hrulefill} \\[3pt]
    학번: \makebox[.3\textwidth]{\hrulefill}

% 점수 추적을 시작
    \addpoints\relax
    %\begin{center}
        %\gradetable[h][questions] 
    %\end{center}

\section*{객관식 문제: (문제당 2점)}

\begin{questions}
\question고용관계의 행위자로서 우리나라 정부에 관한 설명으로서 옳지 않은 것은?
    \begin{choices}
    \choice\relax 정부는 입법이나 행정을 통해 고용관계를 대신하는데, 대표적 사례가 최저임금제도이다.
    \choice\relax 고용관계에서 정부의 역할 비중은 과거에 비해 증가했다.
    \choice\relax 정부는 공공기관 종사자의 사용자로서 민간 기업의 고용관계를 이끌기도 한다.
    \CorrectChoice\relax 정부는 노동문제에 직접적인 영향력을 행사하지 않고 중립적인 위치를 취한다.
    \end{choices}

\question우리나라 노동조합운동의 역사에 관한 설명으로 옳은 것은?
    \begin{choices}
    \choice\relax 이승만 정부 시기 주요 노동쟁의의 원인은 최저임금법 미준수 문제였다.
    \choice\relax 8·15 해방과 함께 미군정이 개시되면서 노동자의 근로조건 개선을 위한 노조 설립과 쟁의 행위가 활성화되었다.
    \choice\relax 4·16 쿠데타 이후 정부가 모든 노동조합을 해체한 이후 사용자가 주도적으로 기업별 노조를 설립했다.
    \CorrectChoice\relax 87년 노동자 대투쟁은 임금·노동조건 개선이라는 경제투쟁이면서 동시에 민주화라는 사회운동 지향성을 갖는 정치투쟁이었다.
    \end{choices}

\question우리나라 노동조합에 대한 설명으로 옳은 것은?
    \begin{choices}
    \CorrectChoice\relax 헌법 제 33조는 노동 3권을 인정하고 있다.
    \choice\relax 기업규모별로 지불능력이 유사하여 산업별 조합이 일찍부터 발달했다.
    \choice\relax 노동조합 설립 시 신고주의를 채택하고 있으며 행정관청이 신고를 반려하면 노조를 만들 수 없다.
    \choice\relax 1987년에 노동조합 가입자수가 폭증하여 정점을 찍은 이후 노조조직률이 하락 추세이다.
    \end{choices}

\question비정규직 노조에 대한 설명으로 옳지 않은 것은?
    \begin{choices}
    \choice\relax 교섭단위와 조직단위가 불일치하는 경우가 많다.
    \CorrectChoice\relax 특수고용형태종사자는 자영업자이기 때문에 노동조합을 만들 수 없다.
    \choice\relax 임금과 근로조건 등을 실질적으로 개선시킬 수 있는 사용자측 당사자를 찾기 어렵다.
    \choice\relax 노동조합을 인정하지 않고 단체교섭을 거부하는 사용자를 상대로 단체교섭을 요구하는 파업을 하는 경우도 있다.
    \end{choices}

\question단체교섭에 관한 설명으로 옳은 것은?
    \begin{choices}
    \choice\relax 산업별 교섭에서 근로자측 교섭 담당자는 단위노조의 대표자 및 교섭위원으로 지명된 조합원이다.
    \choice\relax 전국대학노조와 개별 대학교의 사용자간의 교섭은 통일교섭이라고 할 수 있다.
    \choice\relax 노조는 조합원을 대표하는 조직이므로 사용자와의 교섭이 이루어지기 전에 부서간 내부 조율이 별도로 필요하지 않다.
    \CorrectChoice\relax 단체교섭은 그 자체가 목적이나 귀결점이 아닌 과정이므로 노사간의 전반적인 관계를 형성하고 개선한다.
    \end{choices}

\question쟁의조항에 대한 설명으로 옳은 것은?
    \begin{choices}
    \choice\relax 쟁의행위는 노동권에 포함되므로 노조의 자유의사로 개시할 수 있다.
    \choice\relax 쟁의행위는 노조원이라면 모두가 참여해야 한다.
    \CorrectChoice\relax 쟁의행위중 임시고용된 사람은 쟁의조합원의 부서업무를 대체할 수 없다.
    \choice\relax 쟁의조항은 쟁의의 평화적 해결에 도움이 되기 때문에 노사 양측에서 협약을 찬성한다.
    \end{choices}

%\question직업별 조합에 대한 설명 중 옳지 않은 것은?
    %\begin{choices}
    %\choice\relax 산업화 초기에 등장한 조합 형태이다.
    %\CorrectChoice\relax 노동시장에 대한 통제력이 부족하다.
    %\choice\relax 실업자도 조합에 가입할 수 있다.
    %\choice\relax 전국공무원노조는 직업별 조합이다.
    %\end{choices}

\question단체교섭의 대상에 대한 설명으로 옳은 것은?
    \begin{choices}
    \choice\relax 모든 교섭 사항에 대해 사용자가 거부 또는 성실히 교섭하지 않을 경우 부당노동행위로 간주된다.
    \CorrectChoice\relax 경영권이라 하더라도 근로자의 이해와 직접 관련된 사항은 의무적 교섭사항이다.
    \choice\relax 회사의 인수합병 같은 경영 사항은 단체 교섭 대상이 아니다.
    \choice\relax 근로자의 임금이나 근로조건이 아닌 단체협상에 관한 사항은 임의적 교섭사항이다.
    \end{choices}

\question다음 중 옳지 않은 것은?
    \begin{choices}
    \choice\relax 박정희 정권은 경제 개발을 위해 노동정책을 경제정책에 종속시키고 노동운동을 탄압했다.
    \choice\relax 2011년 이후 노조직률이 상승하고 있는데 복수노조설립을 허용하여 노동자들의 다양한 이해관계가 수용될 수 있는 제도적 조건이 마련되었기 때문이다.
    \CorrectChoice\relax 기업 내 노조가 많으면 단위 노조의 교섭력이 커진다.
    \choice\relax 단체협약이 쌍방 간의 상호이익적 관계를 근간으로 하여 사회적 계약의 형태를 취하는 것은 단체교섭이 통합적 교섭의 요소를 갖고 있기 때문이다.
    \end{choices}

%\question다음 중 옳은 것은?
    %\begin{choices}
    %\choice\relax 2000년대 이후 비정규직을 보호하기 위해 마련된 비정규직 보호법안은 하청업체 근로자에 대한 임금 차별을 해소하기 위한 목적을 가지고 있었다.
    %\CorrectChoice\relax 자본주의 부를 재분배하고 자본주의 체제를 전복시키는 것을 목표로 하여 급진적인 투쟁 방식을 취하는 노동조합의 이데올로기는 혁명적 노동조합주의이다.
    %\choice\relax 경총 (한국경영자총협회)은 전국단위의 유일한 사용자조직으로 주로 재벌이나 대기업의 이해관계를 대변하고 있다.
    %\choice\relax 제 1공화국의 대한노총은 해방 이후 노동자의 사회적 지위와 경제적 복지 향상에 크게 기여했다.
    %\end{choices}

\question다음 중 옳지 않은 것은?
    \begin{choices}
    \choice\relax 단체교섭은 노사가 서로 상반되는 주장에 대하여 다양한 수단과 방법을 동원하여 타결점을 찾으려는 일련의 정치적 과정이다.
    \choice\relax 우리나라의 경우 노사협상을 협의의 단체협상과 임금협상으로 나누어 각각 실시하는 것이 관행이다.
    \choice\relax 단체교섭은 노사간의 전반적인 관계를 개선하기 위한 정서적 요소를 가지고 있다.
    \CorrectChoice\relax 사용자가 특정 지역 내에서 성공적으로 기업을 운영하고 있고 다른 지역으로 이동이 어려울 경우 그 기업의 노동조합의 교섭력은 약하다.
    \end{choices}

\question노동쟁의에 대한 설명으로 옳은 것은?
    \begin{choices}
    \choice\relax 생산 또는 사무를 방해하는 행위는 노동법상 합법적 쟁의 수단이 아니다.
    \choice\relax 직장폐쇄에 대한 저항으로 하는 파업을 공격적 파업이라 한다.
    \choice\relax 1987년 전후 노동쟁의 발생 원인은 임금보다는 일자리 안정 이슈가 중요했다.
    \CorrectChoice\relax 단체협약을 지키지 않았다는 이유로 파업을 하면 불법파업이다.
    \end{choices}

\question쟁의조정에 대한 설명으로 옳지 않은 것은?
    \begin{choices}
    \choice\relax 단체교섭이 결렬되면 당사자 일방의 신청에 의해 조정이 개시된다.
    \choice\relax 필수공익사업장에 정부의 긴급조정이 공표되면 쟁의행위를 즉시 중지하고 30일간 쟁의행위를 금지해야 한다.
    \choice\relax 조정은 강제가 아니므로 조정안을 수락할지 여부는 당사자들의 자유 의사이다.
    \CorrectChoice\relax 중재 제도는 노사 간의 의사소통을 도와 자율적으로 협상하도록 조력하는 제도이다.
    \end{choices}

%\question다음 중 옳지 않은 것은?
    %\begin{choices}
    %\choice\relax 근로시간면제제도 (타임오프제도)는 노조활동을 보장한다.
    %\CorrectChoice\relax 근로기준법에서 정한 연장근로시간 주 최대 12시간은 모든 사업장에 적용된다.
    %\choice\relax 무노조 기업에서 노조가 결성되면 경영이 보다 전문화되는데 이것을 충격 효과라고 한다.
    %\choice\relax 한국전력공사 노동조합은 정부 재정 지원으로 노동비용 인상분을 기업이 자체적으로 흡수할 수 있어 교섭력이 크다.
    %\end{choices}

\question1996년 노동법 개정 가운데 친사용자 조항이 아닌 것은? 
    \begin{choices}
    \choice\relax 노조전임자임금지급금지   
    \choice\relax 무노동·무임금원칙 명문화 
    \CorrectChoice\relax제3자개입금지조항삭제
    \choice\relax 변형근로기간제도입       
    \end{choices}

\question부당노동행위의 구제절차에 대한 설명으로 옳은 것은?
    \begin{choices}
    \choice\relax 노동위원회 심사는 3심제가 원칙이다.
    \CorrectChoice\relax 부당노동행위는 행위발생 3개월 이내에 구제신청이 가능하다.
    \choice\relax 구제명령에 불복한다면 한 달 안에 재심 신청을 할 수 있다.
    \choice\relax 노동위원회의 최종결정은 절대적이며 노동자와 사용자가 모두 따라야 한다.
    \end{choices}
\pagebreak

\section*{빈 칸 채우기: (빈 칸당 2점)}

%\question고용관계는 기업의 \fillin[효율성]{} 제고와 인간의 존엄성과 자유를 발현시킬 수 있는\fillin[공정성]{} 실현을 목적으로 한다.
\question\fillin[분배적 교섭]{}은/는 단체교섭의 구성요소로, 비용측면에서 상대방을 악화시키기 위해 각종 전술을 사용하여 자신의 이익을 극대화 한다. 이는 노사 양측이 함께 만족하는 합의한을 도출하려는 \fillin[통합적 교섭]{}과 반대된다.
\question\fillin[노동위원회]{}은/는 노동관계의 안정과 발전을 위해 조정과 판정 업무를 독립적으로 수행하는 준사법적 기관 이다.
%\question\fillin[노동집약적]{} 산업에서는 생산과정에서 노동력이 매우 중요하고 노조의 파업이 기업운영에 치명적인 영향을 미치므로 사용자의 교섭력은 상대적으로 약하다.
\question\fillin[전태일 사건]{} 1970년대 근로조건의 개선미흡, 산업간·학력간·남녀간의 임금격차 심화 등 근로자의불만 표면화 되는 시기에 발생했다.
\question\fillin[민주노총]{}의 모태는 1990년 1월에 결성된 진보적 성향을 가진 전국노동조합협의회 (전노협)였고, 1999년 전국중앙조직에 대한 복수노조가 허용됨에 따라 합법조직이 되었다.
\question버스업계에서도 1960년대부터 지역별로 다수의 버스사업주와 노조가 각각 교섭대표를 선출하여 \fillin[집단교섭]{} 방식으로 교섭을 하고 있다.
\question여러 노동조합 조직형태 가운데 \fillin[일반조합]{}은/는 전체 노동자를 위한 최저 노동기준의 준수라는 요구 조건의 실현을 위해 입법 활동을 중시한다.
%\question\fillin[사회적 노동조합주의]{}은/는 진보적인 성격이 강한 스웨덴, 덴마크 등 북유럽의 노동조합이 신봉하고 있다.
\question단체교섭을 할 때 \fillin[예비회담]{} 단계에서는 노사가 상호 교섭하고자 하는 항목을 교환한다.
\question\fillin[조정전치주의]{}은/는 조정절차를 거치지 않으면 쟁의행위를 할 수 없도록 하는 것을 말한다.
%\question\fillin[계산적 파업]{} 이/란 상황에 대한 정확한 이해와 목적지향적인 행동에 근거하여 수행되는 파업이다.
%\question단체협약이 성립되면 그 유효기간 중 노사 쌍방이 이를 존중하고 준수할 의무를 지게 되므로 그 기간 중에는 불필요한 분쟁을 피하고 \fillin[산업평화]{}을/를 유지시키는 기능을 하게 된다.
\question유해위험작업종사자의 근로기준시간은 1주간 \fillin[34시간]{}이다.
\question노동쟁의 상의 사용자 대항행위에는 \fillin[직장폐쇄]{}와/과 \fillin[조업계속]{} 이/가 있다.
\question쟁의조정시 토지주택공사는 \fillin[10]{}일간 쟁의행위가 금지된다. 부산은행은 \fillin[15]{}일간 쟁의행위가 금지된다.
%\question\fillin[노란봉투법]{}은/는 단체교섭 또는 쟁의행위로 인하여 손해를 입은 경우에 사용자가 노동조합 또는 근로자에 대하여 그 배상을 청구할 수 없도록 하여, 파업의 범위를 좀 더 넓혀서 합법적인 파업을 할 수 있도록 하고 노조활동을 억압하려는 목적의 손배·가압류 청구를 금지할 목적으로 입법 발의된 법안이다.
%\question문재인 정부는 박근혜 정부가 추진한 노동개혁 가운데 양대 지침인 \fillin[일반해고]{}와/과 \fillin[취업규칙 변경·완화]{} 폐기를 선언했다.

\pagebreak

\section*{약술형 문제}
모든 약술형 답은 \textbf{수업시간에 배운 개념}을 바탕으로 답과 근거를 모두 쓰시오.

\question김수현은 연간 총소득이 1억 2천만원으로 외벌이 이다. 전업주부인 아내와 미성년자인 두 자녀를 가족으로 두고 있으며, 은퇴한 어머니를 함께 모시고 있다. 작년에 아들이 서울로 유학을 갔고 학비 및 생활비로 매달 백만원을 보내고 있다. 김지원은 연간 총소득이 5천만원인데, 배우자는 연간 총소득 3천만원으로 맞벌이 생활을 하고 있다.
\begin{parts}
    \part[5] 올해 평균 가구소득을 7000만원이라고 가정하자. 위에 제시된 두 사람의 가구가 중산층인지 여부를 답하시오.
        \ifprintanswers\relax
            \begin{itemize}
                \item \textbf{설명}: 중산층의 기준은 가구소득의 50--150\%이다. 중산층은 연간 총소득이 3,500만원-1억 500만원 사이인 가구이다. 김수현 가구는 중산층보다 소득이 높고, 김지원 가구는 중산층이다.
                \item \textbf{부분점수}: 중산층 기준 설명하면 2점
            \end{itemize}
        \else 
            \\[3pt]
            \rule{\linewidth}{0.4pt} \\[3pt]
            \rule{\linewidth}{0.4pt} \\[3pt]
            \rule{\linewidth}{0.4pt}
        \fi
    \stepcounter{part}
    \part[10] 소득분배의 관점에서 김수현과 김지원 중 누구의 가구 더 높은 소득수준을 유지하는가? 구체적인 수치로 제시하라. (단, 제곱근은 소수 첫째자리로 계산한다.)
        \ifprintanswers\relax
            \begin{itemize}
                \item \textbf{설명}: 소득분배의 관점에서 비교하려면 균등화처분가능 소득으로 변환해야 한다.
                    \[ 
                        \text{김수현가구 소득} = \frac{1.2\text{억원}-(\text{백만원}\times12)}{\sqrt{4+1-1}} = 5400\text{만원}
                    \]
                    \[ 
                        \text{김지원가구 소득} = \frac{5\text{천만원}+3\text{천만원}}{\sqrt{2}} \approx 5714\text{만원}
                    \]
                    따라서 김지원 가구가 소득분배 관점에서 더 높은 소득수준을 누린다.
                \item \textbf{부분점수}: 계산 시도하면 3점
            \end{itemize}
        \else
            \\[3pt]
            \rule{\linewidth}{0.4pt} \\[3pt]
            \rule{\linewidth}{0.4pt} \\[3pt]
            \rule{\linewidth}{0.4pt} \\[3pt]
            \rule{\linewidth}{0.4pt} \\[3pt]
            \rule{\linewidth}{0.4pt} \\[3pt]
            \rule{\linewidth}{0.4pt}
        \fi
\end{parts}

\question[10] 전공의 집단사직이 사회적으로 문제가 되고 있다. 전공의 협의회에 적합한 조직형태를 제시하고 장·단점을 사례에 맞게 서술하시오.
    \ifprintanswers\relax
        \begin{itemize}
            \item \textbf{설명}: 직업별 노동조합
            \item \textbf{부분점수}: 수업시간에 배운 다른 개념을 사용한 서술은 최대 3점
        \end{itemize}
    \else
        \\[3pt]
        \rule{\linewidth}{0.4pt} \\[3pt]
        \rule{\linewidth}{0.4pt} \\[3pt]
        \rule{\linewidth}{0.4pt} \\[3pt]
        \rule{\linewidth}{0.4pt} \\[3pt]
        \rule{\linewidth}{0.4pt}
    \fi

\question[10] 자본주의와 노사갈등간의 관계에 대하여 설명하시오.
    \ifprintanswers\relax
        \begin{itemize}
            \item \textbf{설명}: 노사갈등에 대한 다원론적 입장
            \item \textbf{부분점수}: 수업시간에 배운 다른 개념을 사용한 서술은 최대 5점
        \end{itemize}
    \else
        \\[3pt]
        \rule{\linewidth}{0.4pt} \\[3pt]
        \rule{\linewidth}{0.4pt} \\[3pt]
        \rule{\linewidth}{0.4pt} \\[3pt]
        \rule{\linewidth}{0.4pt} \\[3pt]
        \rule{\linewidth}{0.4pt}
    \fi

\end{questions}
%------------------------------------------------
\end{document}
%------------------------------------------------
