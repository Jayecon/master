\begin{tabular}{lp{8cm}p{8cm}}
\toprule
& \textbf{공공부문} & \textbf{민간부문} \\
\midrule
\textbf{주요 목적} & 공공성 추구 & 이윤 추구 \\
\midrule
\textbf{관련 기제} & 정부 & 시장 \\
\midrule
\textbf{예산제약} & 공공적 통제 & 경제적 제약 \\
\midrule
\textbf{정치적 성격} & 정치적 성격 강함 & 정치적 성격 약한 편 \\
\midrule
\textbf{사용자} & 사용자불명확성 \newline 경영자 자율성 약함 \newline 사실상의 사용자로서 정부 존재 & 사용자가 명확함 \newline 경영자 자율성 강함 \newline \\
\midrule
\textbf{통제} & 다양한 이해집단에 의한 다중통제 & 기업내 통제 \\
\midrule
\textbf{경쟁} & 국내외 시장 힘으로부터 비교적 자유로움 & 국내외 기업과 경쟁 \\
\midrule
\textbf{목표} & 목표다양성, 목표간 상호모순 가능성 & 비교적 명확하고 단일한 목표 \\
\midrule
\textbf{노사관계} & 갈등관계 또는 유착관계 & 갈등관계 또는 중속관계 \\
\midrule
\textbf{교섭} & 노조/사용자/정부/시민 다면교섭 & 노조/사용자 양자교섭 \\
\midrule
\textbf{갈등} & 기업내 노사 갈등 & 기업내 노사 갈등 \\
\bottomrule
\end{tabular}

