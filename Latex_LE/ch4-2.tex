%----------------------------------------------------------
% PACKAGES AND THEMES
%----------------------------------------------------------
\documentclass[aspectratio=169,xcolor=dvipsnames,handout]{beamer}

\usetheme{Darmstadt}
\usecolortheme{seahorse}

\usepackage[hangul]{kotex}
\usepackage{hyperref}
\usepackage{graphicx, array, adjustbox, makecell}
\usepackage{booktabs, multicol, multirow}
\setbeamercovered{transparent}

%----------------------------------------------------------
% TITLE PAGE
%----------------------------------------------------------
\title{단체협약}
\subtitle{노사관계의 이론과 실제}
\author{오성재}
\institute[CNU]
{\relax
    충남대학교 경제학과\\
}
\date{2024년 9월 30일}

%----------------------------------------------------------
\begin{document}
%----------------------------------------------------------

\frame{\titlepage}

\begin{frame}{목차}
    \small
    %아래 둘 중 하나만 쓸 것
    \tableofcontents[hideallsubsections]
    %\tableofcontents
\end{frame}

\section{단체협약의 개념과 성격}

\begin{frame}[allowframebreaks]
    \frametitle{의의와 개념}
    \begin{block}{단체협약 (collective agreement)}
        노동조합 또는 그 연합체와 사용자 또는 사용자 단체간 체결되는 집단적 근로관계에 관한 계약 
    \end{block}
    \begin{itemize}[<+->]
        \item 단체협약의 의의
        \begin{itemize}[<+->]
            \item 개별근로자와 사용자간의 자유로운 계약은 양자간의 힘의 불균형으로 평등한 계약이 어려움
            \item 개별 근로자들은 단결하여 노동조합을 결성
            \item 집단으로 부터 발생하는 교섭력을 통해 사용자와 대등한 입장에서 교섭을 하여 임금 기타 근로조건 등을 결정 
            \item 이를 문서화된 것이 단체협약
        \end{itemize}
    \framebreak\relax
    \item 단체협약의 내용: 당사자가 체결하는 서면상의 계약
        \begin{itemize}[<+->]
            \item 임금, 근로시간, 기타 근로자의 대우에 관한 사항 
            \item 조합원의 범위, 숍 (shop) 제도 
            \item 조합활동을 위한 절차와 요건
            \item 단체교섭절차, 쟁의행위에 관한 사항 등등
        \end{itemize}
    \end{itemize}
\end{frame}

\begin{frame}
    \frametitle{성격}
    \begin{itemize}[<+->]
    \item 형식적 측면: 노사 양측에 의한 단체적인 약속인 합의
        \begin{itemize}[<+->]
            \item 근로계약: 노사 합의라는 측면에서 유사. 계약 당사자가 개별 근로자라는 점이 상이 
            \item 취업규칙: 계약 당사자 측면에서 유사. 사용자가 사전에 일방적으로 결정했다는 점에서 상이
                \begin{exampleblock}{노동조합법 제 33조 제 1항}
                    단체협약에 정한 근로조건 기타 근로자의 대우에 관한 기준에 위반하는 취업규칙 또는 근로계약의 부분은 무효로 한다
                \end{exampleblock}
        \end{itemize}
    \item 노사관계적 측면: 노사간의 일시적인 합의. 즉, 휴전조약과 같은 성격
        \begin{itemize}[<+->]
            \item 대립적 고용관계에서 발생하는 분쟁의 소지를 해소
            \item 단체협약을 통해 약속된 기간동안 산업의 평화를 유지
        \end{itemize}
    \end{itemize}
\end{frame}

\begin{frame}
    \frametitle{기능}
    \begin{itemize}[<+->]
        \item 집단노동관계의 당사자인 노동조합과 사용자의 행동을 규제하는 자체 규범
        \begin{itemize}[<+->]
            \item 협약의 성실한 이행과 운용이 필요
        \end{itemize}
        \begin{enumerate}[<+->]
            \item 근로조건 개선기능: 근로자에 대한 노동조건의 개선기능
            \begin{itemize}[<+->]
                \item 개별 근로자와 사용자간의 관계에서 성립될 수 있는 것보다 훨씬 유리한 근로조건을 단체협약을 통해 확보 가능
                \item 이는 노동조합이 단결권 및 쟁의행동권을 바탕으로 강한 교섭력을 갖고 있기 때문
            \end{itemize}
            \item 산업평화 기능: 단체협약 유효기간동안 노사 쌍방이 평화의무 준수
            \begin{itemize}[<+->]
                \item 단체협약의 평화기능: 단체협약의 유효기간 동안 무의미한 분쟁을 피하고 협약내용을 성실히 준수할 의무를 가짐
            \end{itemize}
        \end{enumerate}
    \end{itemize}
\end{frame}

\section{단체협약의 성립과 관리}

\begin{frame}
    \frametitle{단체협약의 당사자 및 작성방식}
    \begin{itemize}[<+->]
        \item 단체협약의 당사자
        \begin{itemize}[<+->]
            \item 협약능력을 가진 자로서 노동조합과 사용자 또는 사용자 단체
        \end{itemize}
        \item 단체협약의 작성방식
        \begin{itemize}[<+->]
            \item 서면으로 작성하여 당사자 쌍방이 서명·날인
            \item 단체협약 체결일로부터 15일 이내에 행정관청에 당사자 쌍방의 연명으로 신고
            \item 행정관청은 단체협약의 내용 중에 위법한 내용이 있는 경우 노동위원회의 의결을 얻어 그 시정을 명할 수 있음
        \end{itemize}
    \end{itemize}
\end{frame}

\begin{frame}
    \frametitle{단체협약의 유효기간}
    \begin{itemize}[<+->]
        \item 단체협약의 유효기간
        \begin{itemize}[<+->]
            \item 단체협약이 3년을 초과하는 유효기간을 정할 수 없음
            \item 유효기간을 정하지 않았거나 3년을 초과하여 정할 경우 유효기간은 3년이 됨
            \item 새로운 단체협약을 체결하지 못할 경우 구 단체협약이 3개월 유효 연장.
            \item 단, 임금협약의 유효기간을 1년으로 약정하는 것은 유효함.
        \end{itemize}
    \end{itemize}
    
\end{frame}

\section{단체협약의 내용}

\begin{frame}[allowframebreaks]
    \frametitle{노조안정과 경영권}
    \begin{itemize}[<+->]
        \item 노조: 활동의 안정과 보장을 위해 숍제도, 노조활동 등을 요구
        \item 사용자: 경영권보호에 집중
    \end{itemize}
    \begin{enumerate}[<+->]
    \item 숍제도
        \begin{itemize}[<+->]
            \item 비조합원의 존재로 인한 조합의 불이익을 막기 위한 조치
            \item union shop: 근로자의 3분의 2 이상을 대표하는 노동조합에게 인정하는 숍제도로서 약 36.9\% 규정 
            \item open shop: 단체협약에 약 56\% 규정하고 있음 
        \end{itemize}
    \framebreak\relax
    \item 노조활동 조항: 노조활동 보장, 근무시간 중 조합활동, 노조 전임자, 조합비 공제편의제공 등등
        \begin{itemize}[<+->]
            \item 조합활동의 보장조항: 헌법 제31조 및 「노동조합 및 노동관계조정법』에 의해 보장되어 있음
            \begin{itemize}[<+->]
                \item 대부분의 단체협약에서 `조합활동의 자유 (및 불이익처우의 금지)' 등의 규정으로 설정
            \end{itemize}
            \item 근로시간 중 조합활동조항: 기업별 노조의 경우 노무 제공과 정당한 노조활동이 중복될 수 있으므로 이에 대한 규정 설정 필요
            \item 노조전임자 규정: 전임자의 필요성으로 생긴 조항으로 전임자 수, 인선 시 간섭 배제, 전임중의 급여, 노동조건, 전임 소멸 후 직장복귀, 전임시간 등에 대한 규정 설정
            \item 조합비공제편의제공조항: check-off system에 대한 규정
        \end{itemize}
    \item 경영권: 회사운영과 노동력 지휘 등 사용자가 가지는 전통적인 기능 (인사권, 관리권 등등)
        \begin{itemize}[<+->]
            \item 경영권은 사용자의 고유 권한이므로 조합원이 간섭할 수 없도록 규정하는 경우가 많음
        \end{itemize}
    \end{enumerate}
\end{frame}

\begin{frame}
    \frametitle{보수}
    \begin{itemize}[<+->]
        \item 임금조항: 임금조항을 규정하고 있는 협약이 적음
        \begin{itemize}[<+->]
            \item 대체로 임금협정이 별도로 이루어지기 때문에 임금조항은 상대적으로 적음 
            \item 임금공제항목, 비상지불조항, 휴업수당 등에 관한 조항 등이 있음
            \item 상여금 및 퇴직금 지급기준 등
        \end{itemize}
    \end{itemize}
\end{frame}

\begin{frame}
    \frametitle{인사조항}
    \begin{itemize}[<+->]
        \item 노동조합측: 인사권에 대한 영향력 강화 요구
        \item 사용자측: 인사권은 경영권
        \begin{itemize}[<+->]
            \item 사용자측의 방침이나 조합의 관심 등에 따라 인사조항에 차이가 발생
        \end{itemize}
    \end{itemize}
    \begin{enumerate}[<+->]
        \item 수습기간: 일정한 수습기간을 규정하는 협약
        \item 해고협의: 해고는 조합의 동의 (또는 협의)를 요한다는 식의 협의조항
        \begin{itemize}[<+->]
            \item 해고협의 조항은 해고권에 관한 자율적 제약을 확립한 것이기 때문에 이 조항을 위반한 해고는 법적으로는 무효.
            \begin{itemize}[<+->]
                \item 노사가 동의한 약정을 위반한 해고는 무효
            \end{itemize}
            \item 이 조항에서 동의는 노동조합의 해고에 대한 실질적인 동의를 의미.
            \begin{itemize}[<+->]
                \item 노사가 타결점을 발견하기 위해 성실한 심의를 수행. 
                \item 반드시 합의점에 도달할 필요는 없음.
            \end{itemize}
        \end{itemize}
    \end{enumerate}
\end{frame}

\begin{frame}
    \frametitle{복리후생조항}
    \begin{itemize}[<+->]
        \item 근로자복지제도는 법정복지제도와 법정 외 복지제도로 구분할 수 있음
        \item 법정복지제도
        \begin{itemize}[<+->]
            \item 건강보험, 연금보험, 고용보험, 산재보험
        \end{itemize}
        \item 법정 외 복지제도 임금, 보험 이외에 기업이 제공하는 모든 편익을 의미함
            \begin{itemize}[<+->]
                \item 주거비지원, 보건의료지원, 식비지원, 문화체육활동지원, 경조비지원, 보육비지원
            \end{itemize}
        \end{itemize}
\end{frame}

\begin{frame}
    \frametitle{산업안전보건 조항}
    \begin{itemize}[<+->]
        \item OECD 평균 대비 산업재해의 대형화·심각화, 직업병의 이환 (발병)가능성 증대
        \begin{itemize}[<+->]
            \item 노조의 작업안전 및 보건 등에 관심 요구됨
        \end{itemize}
    \item 노동조합: 작업환경 및 시설의 개선, 임금노동조건의 개선, 관리체계의 합리화, 보건에 대한 교육기회의 확대 등을 요구
        \begin{itemize}[<+->]
            \item 산업안전보건위원회 구성
            \item 작업환경 측정
            \item 건강진단
        \end{itemize}
    \end{itemize}
\end{frame}

\begin{frame}[allowframebreaks]
    \frametitle{근로시간과 교대제}
    \begin{itemize}
        \item 근로시간 
        \begin{itemize}[<+->]
            \item 실무시간: 사용자의 지휘명령 하에서 노무를 제공하는 시간
            \item 구속시간: 노무의 제공에서 벗어나는 휴식시간을 포함한 시간
            \item 『근로기준법』 「산업안전보건법』에서 정한 근로기준시간
            \begin{itemize}[<+->]
                \item 일반 성인근로자: 1주일 40시간
                \item 유해위험작업 종사자: 1일 6시간, 1주 34시간 
                \item (15--18세) 연소근로자: 1일 7시간, 1주 40시간
                \item 연장근로시간은 주 최대 12시간
            \end{itemize}
        \end{itemize}
    \framebreak\relax
    \item 교대제
        \begin{itemize}[<+->]
            \item 근무제도는 인력, 근로시간, 법정수당과 긴밀하게 연관되어 있음 
            \item 단체교섭에서 논의되는 주된 내용임
            \begin{itemize}[<+->]
                \item 사업체의 운영시간이 일일 근로시간 (보통 8시간)보다 길 경우 교대제를 운영 
                \item 예: 2조 격일제, 2조2교대제, 주간연속 2교대제, 3조2교대제, 4조3교대제 등등
                \item 2조 격일제와 2조2교대제가 보편적으로 많이 활용되며 장시간 근로의 근무형태
            \end{itemize}
        \end{itemize}
    \end{itemize}
\end{frame}

\begin{frame}[allowframebreaks]
    \frametitle{단체교섭·쟁의·노사협의제}
    \begin{itemize}[<+->]
        \item 단체교섭조항을 별도로 규정
        \begin{itemize}[<+->]
            \item 교섭 실시에 따른 절차·운영에 대한 사항 사전 정하기 위해
            \item 내용: 사전요청서 제출규정, 협약의 갱신, 운영에 관한 사항, 단체교섭의 인원수 및 선임방법 등등
        \end{itemize}
        \item 노사협의제 조항: 『근로자 참여 및 협력에 관한 법률』 제정으로 노사협의제를 별도로 규율
        \begin{itemize}[<+->]
            \item 노사협의회 합의사항의 준수 및 이행을 단체협약에서 재차 규정 (54.4\% 규정)
        \end{itemize}
    \framebreak\relax
        \item 쟁의조항: 쟁의행위를 감행할 때의 규범을 정한 협약조항
        \begin{itemize}[<+->]
            \item 예고조항: 쟁의행위 실시 수일 전 상대방에게 통고
            \item 불참가자조항: 쟁의행위에 참가하지 않는 사람의 범위 규정 (예: 보안요원 등)
            \item 대체근로 금지조항: 쟁의 중 임시 고용된 사람이 쟁의조합원 부서업무를 대체하지 못한다는 규정
            \item 평화의무조항: 분쟁을 평화적 교섭으로 해결하기 위해 쟁의행위 개시, 요건 및 제한 등의 규정
        \end{itemize}
        \item 노조측는 쟁의행위 억제효과가 있으므로 반대, 사용자측은 동일한 이유에서 찬성하는 경향
    \end{itemize}
\end{frame}

%------------------------------------------------
\end{document}
%------------------------------------------------
