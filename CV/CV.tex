\documentclass[11pt,a4paper,sans]{moderncv}
\moderncvstyle{classic}
\moderncvcolor{blue}
\usepackage[utf8]{inputenc}
\usepackage{kotex}

\name{오성재}{}
\title{이력서}
\address{대전시 서구 관저로 83, 213동 1303호}{35372}{}
%\address{address}{zipcode}{country}
\phone[mobile]{010--3108--8775}
\email{sungjae81@gmail.com}

\begin{document}
\raggedright\relax
\makecvtitle\relax

\section{학력}
%\cventry{시작연도--종료연도}{학위 또는 직책}{기관}{도시}{상세 정보}{설명}
\cventry{2015--2022}{경제학박사}{서울대학교}{}{지도교수: 주병기}{학위논문명: 한국의 경제 및 교육의 기회불평등 분석 \\ 심사위원: 김봉근, 주병기, 홍석철, 강창희, 이우진}
\cventry{2008--2012}{경제학석사}{고려대학교}{}{지도교수: 주병기}{학위논문명: A Simple Characterization of Majority Rule on Dichotomus Preferences.}
\cventry{2001--2008}{경제학사}{고려대학교}{}{}{}

\section{경력}
\cventry{2024--현재}{부연구위원}{한국보건사회연구원}{}{}{}
\cventry{2024--2024}{강사}{충남대학교 경제학과}{}{}{\footnotesize강의명: 노사관계의 이론과 실재.\ \\ 강의내용: 노동조합을 중심으로 사용자 및 정부와의 관계 및 주요 행위를 체계적으로 분석하고 한국의 노동조합의 주요 이슈를 분석한다.}
\cventry{2023--2024}{전임연구원}{SSK 포용적경제정책 연구팀, 전북대학교}{}{}{\footnotesize주요활동: 2건의 학술연구를 완성하였고, 논문게재 1건 및 심사중 1건이 있음.\ 컨퍼런스 준비 등의 연구행정보조 역시 수행.}
\cventry{2022--2022}{초빙교수}{한남대학교 탈메이지 교양대학}{}{}{\footnotesize강의명: 경제정의와 불평등.\ \\ 강의내용: 분배적 정의 (distributive justice)의 문제를 공리주의, 롤즈주의, 자유주의의 입장에서 살펴보고 우리사회 주요 현안에 적용해본다.}
\cventry{2013--2014}{위촉연구원}{KDI 국제정책대학원}{}{}{\footnotesize대중서 \textbf{한국의 경제기적 지난 50년 향후 50년} 작성에 참여. 주요 업무는 표, 그림 작성 및 일부 소절의 초고 작성.}
\cventry{2012--2012}{연구원}{한국개발연구원}{}{}{\footnotesize기관의 세종시 이전으로 연구팀은 KDI 국제정책대학원 소속으로 재배치.}

\section{관심분야 }
%\cvitem{카테고리}{내용}
\cvitem{}{응용미시, 교육경제학, 정책평가}

\section{연구 실적}
\subsection{논문}
\cvitem{1}{\relax
    Sungjae Oh, Hanol Lee.\ (2024).\ Can mandatory local-talent hiring policy reduce regional starting wage gap? Causal evidence from Korean graduates.\ \textit{Economic Analysis and Policy}, 84, 208--229.
    \footnotesize
    \begin{itemize}
    \item 주요내용: 지방공공기관의 지역인재 의무채용 정책이 서울과 지방의 대졸자 초임에 미치는 영향을 분석했다. 2010년부터 2019년까지의 미시자료를 사용한 이중차분 분석 결과, 해당 정책이 비수도권 대졸자의 임금을 상승시켰으며 특히, 인문계열 졸업생에게 특히 더 큰 효과가 나타났다.
    \item 역할: 원문 작성, 검토 및 편집, 시각화, 검증, 방법론, 조사, 정밀 분석, 데이터 관리, 개념 구상
    \end{itemize}
}
\cvitem{2}{\relax
    오성재, 주병기.\ (2017). 한국의 소득기회불평등에 대한 연구.\ \textbf{재정학연구}, 10(3), 1--30.
    \footnotesize
    \begin{itemize}
    \item 주요내용: 가구소득에 대한 기회불평등 분석을 시행한 결과, 가구주의 학력과 소득에 따라 한국에서도 소득의 기회불평등이 나타났다. 이는 미국, 프랑스, 이탈리아와 유사하며, 스웨덴, 노르웨이, 독일과는 차이를 보였다.
    \item 역할: 원문 작성, 검토 및 편집, 시각화, 검증, 방법론, 조사, 정밀 분석, 데이터 관리, 개념 구상
    \end{itemize}
}
\cvitem{3}{\relax
    오성재, 강창희, 정혜원, 주병기.\ (2016).\ 가구환경과 교육성취의 기회: 대학수학능력시험 성적을 이용한 연구, \textbf{재정학연구}, 9(4), 1--32.
    \footnotesize
    \begin{itemize}
    \item 주요내용:  가구의 소득과 남성 보호자의 학력을 기준으로 수능 성적에서 기회불평등을 분석한 결과, 언어와 외국어 평가 영역에서 기회불평등이 존재했으며, 특히 외국어 영역에서 더 두드러졌다. 또한 노력에 따라 기회불평등이 감소하는 경향이 나타났다.
    \item 역할: 검토 및 편집, 시각화, 검증,  정밀 분석, 데이터 관리 구상
    \end{itemize}
}

\subsection{학술 서적}
\cvitem{1}{\relax
    Changhui Kang, Sungjae Oh.\ (2019).\ An International Comparison of Inequality of Educational Opportunity Using TIMSS.\ In M.\ Hosoe, B.\--G.\ Ju, A.\ Yakita \& K.\ Hong (Eds.), \textit{Contemporary Issues in Applied Economics}.\ (pp. 101--119). Springer.
    \footnotesize
    \begin{itemize}
    \item 주요내용: 한국은 사교육 시장 확대로 교육기회불평등이 악화될 가능성이 있다. 1995년부터 2015년까지 TIMSS 자료를 분석한 결과, 한국의 교육기회불평등은 국제적으로 중간 수준에 속하며, 2000년대 중반 이후 불평등이 감소하는 경향을 보였다.
    \item 역할: 검토 및 편집, 시각화, 검증, 방법론, 조사, 정밀 분석, 데이터 관리, 개념 구상
    \end{itemize}
}
\cvitem{2}{\relax
    강창희, 오성재.\ (2017).\ \textbf{기회불평등과 경제성장}.\ 황수경 외 (편), 소득분배와 경제성장 (pp. 185--224). 한국개발연구원.
    \footnotesize
    \begin{itemize}
    \item 주요내용: 학업 성취도 불평등을 기회불평등과 노력불평등으로 구분하여 각각이 경제성장에 미치는 영향을 분석하였다. OECD 가입국에서는 기회불평등이 경제성장에 부정적인 영향을 미친 반면, 노력불평등은 긍정적인 영향을 미쳤다. 이는 기회불평등을 완화하는 정책이 경제의 효율성과 공평성 향상에 기여할 수 있음을 시사한다.
    \item 역할:  검토 및 편집, 시각화, 검증, 조사, 정밀 분석, 데이터 관리, 개념 구상
    \end{itemize}
}
\subsection{심사중 논문}
\cvitem{1}{\relax
    Sungjae Oh, Hanol Lee. High School Reputation and College Admissions in Korea.
    \footnotesize
    \begin{itemize}
    \item 주요내용: 특목고 출신 학생들은 일반 고등학교 출신 학생들에 비해 수시전형에서 출신고교 유형에 의한 후광효과를 누릴 수 있다. 이 연구는 입시전형의 유형과 출신고교의 유형으로 이중차분 분석을 진행하여, 효과의 존재와 그 정도를 실증적으로 제시한다.
    \item 역할: 원문 작성, 검토 및 편집, 시각화, 검증, 방법론, 조사, 정밀 분석, 데이터 관리, 개념 구상
    \end{itemize}
}
\subsection{진행중 연구}
\cvitem{1}{\relax
    Sungjae Oh, Hanol Lee. The Impact of mandatory local-talent hiring policy on Intergenerational Income Mobility.
}
\cvitem{2}{\relax
    Sungjae Oh, Hanol Lee. Machine Learning Based Identification Strategy of Circumstances in the Analysis of Inequality of Opportunity.
}
\cvitem{3}{\relax
    홍현우, 오성재. 공공조달이 사회적기업의 성과에 미친 영향에 대한 연구 (가제)
}

\section{발표}
%\cvitem{카테고리}{내용}
\cvitem{2024. 8.}{East Asia Game Throry 2024, Jeju}
\cvitem{2024. 2.}{경제학공동 학술대회, 서울대학교}
\cvitem{2023. 11.}{경제통상학회, 군산대학교}
\cvitem{2023. 7.}{2023 KER International Conference, Yeosu}
\cvitem{2023. 6.}{2023 고용패널조사 학술대회, 서울대 교수회관}

\section{수상 경력 및 장학금}
\cvitem{2017}{재정학연구 최우수 논문상, 한국재정학회}
\cvitem{2015--2017}{BK21 장학생}
\cvitem{2008--2009}{BK21 장학생}

\section{특기}
\cvitem{프로그램}{Stata, Python, LaTeX, QGIS, Vim}
\cvitem{언어}{한국어, 영어 }

\end{document}

